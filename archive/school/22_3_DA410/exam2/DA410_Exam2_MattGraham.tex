% Options for packages loaded elsewhere
\PassOptionsToPackage{unicode}{hyperref}
\PassOptionsToPackage{hyphens}{url}
%
\documentclass[
]{article}
\usepackage{amsmath,amssymb}
\usepackage{lmodern}
\usepackage{iftex}
\ifPDFTeX
  \usepackage[T1]{fontenc}
  \usepackage[utf8]{inputenc}
  \usepackage{textcomp} % provide euro and other symbols
\else % if luatex or xetex
  \usepackage{unicode-math}
  \defaultfontfeatures{Scale=MatchLowercase}
  \defaultfontfeatures[\rmfamily]{Ligatures=TeX,Scale=1}
\fi
% Use upquote if available, for straight quotes in verbatim environments
\IfFileExists{upquote.sty}{\usepackage{upquote}}{}
\IfFileExists{microtype.sty}{% use microtype if available
  \usepackage[]{microtype}
  \UseMicrotypeSet[protrusion]{basicmath} % disable protrusion for tt fonts
}{}
\makeatletter
\@ifundefined{KOMAClassName}{% if non-KOMA class
  \IfFileExists{parskip.sty}{%
    \usepackage{parskip}
  }{% else
    \setlength{\parindent}{0pt}
    \setlength{\parskip}{6pt plus 2pt minus 1pt}}
}{% if KOMA class
  \KOMAoptions{parskip=half}}
\makeatother
\usepackage{xcolor}
\usepackage[margin=1in]{geometry}
\usepackage{color}
\usepackage{fancyvrb}
\newcommand{\VerbBar}{|}
\newcommand{\VERB}{\Verb[commandchars=\\\{\}]}
\DefineVerbatimEnvironment{Highlighting}{Verbatim}{commandchars=\\\{\}}
% Add ',fontsize=\small' for more characters per line
\usepackage{framed}
\definecolor{shadecolor}{RGB}{248,248,248}
\newenvironment{Shaded}{\begin{snugshade}}{\end{snugshade}}
\newcommand{\AlertTok}[1]{\textcolor[rgb]{0.94,0.16,0.16}{#1}}
\newcommand{\AnnotationTok}[1]{\textcolor[rgb]{0.56,0.35,0.01}{\textbf{\textit{#1}}}}
\newcommand{\AttributeTok}[1]{\textcolor[rgb]{0.77,0.63,0.00}{#1}}
\newcommand{\BaseNTok}[1]{\textcolor[rgb]{0.00,0.00,0.81}{#1}}
\newcommand{\BuiltInTok}[1]{#1}
\newcommand{\CharTok}[1]{\textcolor[rgb]{0.31,0.60,0.02}{#1}}
\newcommand{\CommentTok}[1]{\textcolor[rgb]{0.56,0.35,0.01}{\textit{#1}}}
\newcommand{\CommentVarTok}[1]{\textcolor[rgb]{0.56,0.35,0.01}{\textbf{\textit{#1}}}}
\newcommand{\ConstantTok}[1]{\textcolor[rgb]{0.00,0.00,0.00}{#1}}
\newcommand{\ControlFlowTok}[1]{\textcolor[rgb]{0.13,0.29,0.53}{\textbf{#1}}}
\newcommand{\DataTypeTok}[1]{\textcolor[rgb]{0.13,0.29,0.53}{#1}}
\newcommand{\DecValTok}[1]{\textcolor[rgb]{0.00,0.00,0.81}{#1}}
\newcommand{\DocumentationTok}[1]{\textcolor[rgb]{0.56,0.35,0.01}{\textbf{\textit{#1}}}}
\newcommand{\ErrorTok}[1]{\textcolor[rgb]{0.64,0.00,0.00}{\textbf{#1}}}
\newcommand{\ExtensionTok}[1]{#1}
\newcommand{\FloatTok}[1]{\textcolor[rgb]{0.00,0.00,0.81}{#1}}
\newcommand{\FunctionTok}[1]{\textcolor[rgb]{0.00,0.00,0.00}{#1}}
\newcommand{\ImportTok}[1]{#1}
\newcommand{\InformationTok}[1]{\textcolor[rgb]{0.56,0.35,0.01}{\textbf{\textit{#1}}}}
\newcommand{\KeywordTok}[1]{\textcolor[rgb]{0.13,0.29,0.53}{\textbf{#1}}}
\newcommand{\NormalTok}[1]{#1}
\newcommand{\OperatorTok}[1]{\textcolor[rgb]{0.81,0.36,0.00}{\textbf{#1}}}
\newcommand{\OtherTok}[1]{\textcolor[rgb]{0.56,0.35,0.01}{#1}}
\newcommand{\PreprocessorTok}[1]{\textcolor[rgb]{0.56,0.35,0.01}{\textit{#1}}}
\newcommand{\RegionMarkerTok}[1]{#1}
\newcommand{\SpecialCharTok}[1]{\textcolor[rgb]{0.00,0.00,0.00}{#1}}
\newcommand{\SpecialStringTok}[1]{\textcolor[rgb]{0.31,0.60,0.02}{#1}}
\newcommand{\StringTok}[1]{\textcolor[rgb]{0.31,0.60,0.02}{#1}}
\newcommand{\VariableTok}[1]{\textcolor[rgb]{0.00,0.00,0.00}{#1}}
\newcommand{\VerbatimStringTok}[1]{\textcolor[rgb]{0.31,0.60,0.02}{#1}}
\newcommand{\WarningTok}[1]{\textcolor[rgb]{0.56,0.35,0.01}{\textbf{\textit{#1}}}}
\usepackage{graphicx}
\makeatletter
\def\maxwidth{\ifdim\Gin@nat@width>\linewidth\linewidth\else\Gin@nat@width\fi}
\def\maxheight{\ifdim\Gin@nat@height>\textheight\textheight\else\Gin@nat@height\fi}
\makeatother
% Scale images if necessary, so that they will not overflow the page
% margins by default, and it is still possible to overwrite the defaults
% using explicit options in \includegraphics[width, height, ...]{}
\setkeys{Gin}{width=\maxwidth,height=\maxheight,keepaspectratio}
% Set default figure placement to htbp
\makeatletter
\def\fps@figure{htbp}
\makeatother
\setlength{\emergencystretch}{3em} % prevent overfull lines
\providecommand{\tightlist}{%
  \setlength{\itemsep}{0pt}\setlength{\parskip}{0pt}}
\setcounter{secnumdepth}{-\maxdimen} % remove section numbering
\ifLuaTeX
  \usepackage{selnolig}  % disable illegal ligatures
\fi
\IfFileExists{bookmark.sty}{\usepackage{bookmark}}{\usepackage{hyperref}}
\IfFileExists{xurl.sty}{\usepackage{xurl}}{} % add URL line breaks if available
\urlstyle{same} % disable monospaced font for URLs
\hypersetup{
  pdftitle={DA410\_Exam2\_MattGraham},
  hidelinks,
  pdfcreator={LaTeX via pandoc}}

\title{DA410\_Exam2\_MattGraham}
\author{}
\date{\vspace{-2.5em}}

\begin{document}
\maketitle

\begin{Shaded}
\begin{Highlighting}[]
\FunctionTok{library}\NormalTok{(nnspat)  }\CommentTok{\# used for dist2full()}
\FunctionTok{library}\NormalTok{(}\StringTok{"dplyr"}\NormalTok{)  }\CommentTok{\# used to select numeric datatypes}
\FunctionTok{library}\NormalTok{(}\StringTok{"ggplot2"}\NormalTok{)}
\FunctionTok{library}\NormalTok{(reshape)  }\CommentTok{\# used for melting matricies}
\FunctionTok{library}\NormalTok{(klaR)}
\FunctionTok{library}\NormalTok{(ggvis)}
\FunctionTok{library}\NormalTok{(class)}
\FunctionTok{library}\NormalTok{(gmodels)}
\FunctionTok{library}\NormalTok{(MASS)}
\FunctionTok{library}\NormalTok{(readxl)}
\FunctionTok{library}\NormalTok{(psych)}
\FunctionTok{library}\NormalTok{(corrplot)}
\end{Highlighting}
\end{Shaded}

\begin{verbatim}
## Warning: package 'corrplot' was built under R version 4.2.2
\end{verbatim}

\begin{Shaded}
\begin{Highlighting}[]
\FunctionTok{library}\NormalTok{(lavaan)}
\end{Highlighting}
\end{Shaded}

\begin{verbatim}
## Warning: package 'lavaan' was built under R version 4.2.2
\end{verbatim}

\begin{Shaded}
\begin{Highlighting}[]
\FunctionTok{library}\NormalTok{(semPlot)}
\end{Highlighting}
\end{Shaded}

\begin{verbatim}
## Warning: package 'semPlot' was built under R version 4.2.2
\end{verbatim}

\begin{Shaded}
\begin{Highlighting}[]
\FunctionTok{library}\NormalTok{(semTable)}
\end{Highlighting}
\end{Shaded}

\begin{verbatim}
## Warning: package 'semTable' was built under R version 4.2.2
\end{verbatim}

\begin{Shaded}
\begin{Highlighting}[]
\FunctionTok{library}\NormalTok{(kutils)}
\end{Highlighting}
\end{Shaded}

\begin{verbatim}
## Warning: package 'kutils' was built under R version 4.2.2
\end{verbatim}

\hypertarget{problem-1}{%
\subsection{Problem 1}\label{problem-1}}

Get data

\begin{Shaded}
\begin{Highlighting}[]
\NormalTok{cov.mat }\OtherTok{\textless{}{-}} \FunctionTok{data.frame}\NormalTok{(}\FunctionTok{c}\NormalTok{(}\DecValTok{5}\NormalTok{, }\DecValTok{0}\NormalTok{, }\DecValTok{0}\NormalTok{), }\FunctionTok{c}\NormalTok{(}\DecValTok{0}\NormalTok{, }\DecValTok{9}\NormalTok{,}\DecValTok{0}\NormalTok{), }\FunctionTok{c}\NormalTok{(}\DecValTok{0}\NormalTok{,}\DecValTok{0}\NormalTok{,}\DecValTok{9}\NormalTok{))}
\NormalTok{cov.mat}
\end{Highlighting}
\end{Shaded}

\begin{verbatim}
##   c.5..0..0. c.0..9..0. c.0..0..9.
## 1          5          0          0
## 2          0          9          0
## 3          0          0          9
\end{verbatim}

\hypertarget{a-find-eigenvalues-and-vectors}{%
\subsubsection{A) Find eigenvalues and
vectors}\label{a-find-eigenvalues-and-vectors}}

\begin{Shaded}
\begin{Highlighting}[]
\NormalTok{cov.mat.vals }\OtherTok{\textless{}{-}} \FunctionTok{eigen}\NormalTok{(cov.mat)}\SpecialCharTok{$}\NormalTok{values}
\NormalTok{cov.mat.vals}
\end{Highlighting}
\end{Shaded}

\begin{verbatim}
## [1] 9 9 5
\end{verbatim}

\begin{Shaded}
\begin{Highlighting}[]
\NormalTok{cov.mat.vects }\OtherTok{\textless{}{-}} \FunctionTok{eigen}\NormalTok{(cov.mat)}\SpecialCharTok{$}\NormalTok{vectors}
\NormalTok{cov.mat.vects}
\end{Highlighting}
\end{Shaded}

\begin{verbatim}
##      [,1] [,2] [,3]
## [1,]    0    0    1
## [2,]    0    1    0
## [3,]    1    0    0
\end{verbatim}

\hypertarget{b-find-variance-explained}{%
\subsubsection{B) Find variance
explained}\label{b-find-variance-explained}}

\begin{Shaded}
\begin{Highlighting}[]
\ControlFlowTok{for}\NormalTok{ (r }\ControlFlowTok{in}\NormalTok{ cov.mat.vals) \{}
  \FunctionTok{print}\NormalTok{(r}\SpecialCharTok{/}\FunctionTok{sum}\NormalTok{(cov.mat.vals))}
\NormalTok{\}}
\end{Highlighting}
\end{Shaded}

\begin{verbatim}
## [1] 0.3913043
## [1] 0.3913043
## [1] 0.2173913
\end{verbatim}

We can see that of our eigen values, \textasciitilde80\% of our variance
is explained with 2 dimensions, while \textasciitilde20\% is explained
with 1 dimension. This can be seen below.

\hypertarget{plot}{%
\paragraph{Plot}\label{plot}}

\begin{Shaded}
\begin{Highlighting}[]
\FunctionTok{plot}\NormalTok{(cov.mat.vals}\SpecialCharTok{/}\FunctionTok{sum}\NormalTok{(cov.mat.vals), }\AttributeTok{xlab =} \StringTok{\textquotesingle{}Number of components\textquotesingle{}}\NormalTok{, }\AttributeTok{ylab=}\StringTok{\textquotesingle{}Eigen size\textquotesingle{}}\NormalTok{, }\AttributeTok{main=}\StringTok{\textquotesingle{}Plot of dimension variance\textquotesingle{}}\NormalTok{)}
\end{Highlighting}
\end{Shaded}

\includegraphics{DA410_Exam2_MattGraham_files/figure-latex/unnamed-chunk-5-1.pdf}

\#\#\#C) Decision Ultimately, we will want to select 2 components in our
analysis.

\hypertarget{problem-2}{%
\subsection{Problem 2}\label{problem-2}}

Assumption check:

Variables used should be metric. Dummy variables can also be considered,
but only in special cases. -\textgreater{} check

Sample size: Sample size should be more than 200. -\textgreater{} check

Homogeneous sample: A sample should be homogenous. Violation of this
assumption increases the sample size as the number of variables
increases. Reliability analysis is conducted to check the homogeneity
between variables.

Correlation: At least 0.30 correlations are required between the
research variables.

\begin{Shaded}
\begin{Highlighting}[]
\NormalTok{french }\OtherTok{\textless{}{-}} \FunctionTok{c}\NormalTok{(}\DecValTok{1}\NormalTok{, .}\DecValTok{44}\NormalTok{, .}\DecValTok{41}\NormalTok{, .}\DecValTok{29}\NormalTok{, .}\DecValTok{33}\NormalTok{, .}\DecValTok{25}\NormalTok{)}
\NormalTok{english }\OtherTok{\textless{}{-}} \FunctionTok{c}\NormalTok{(.}\DecValTok{44}\NormalTok{, }\DecValTok{1}\NormalTok{, .}\DecValTok{35}\NormalTok{, .}\DecValTok{35}\NormalTok{, .}\DecValTok{32}\NormalTok{, .}\DecValTok{33}\NormalTok{)}
\NormalTok{history }\OtherTok{\textless{}{-}} \FunctionTok{c}\NormalTok{(.}\DecValTok{41}\NormalTok{, .}\DecValTok{35}\NormalTok{, }\DecValTok{1}\NormalTok{, .}\DecValTok{16}\NormalTok{, .}\DecValTok{19}\NormalTok{, .}\DecValTok{18}\NormalTok{)}
\NormalTok{arithmetic }\OtherTok{\textless{}{-}} \FunctionTok{c}\NormalTok{(.}\DecValTok{29}\NormalTok{, .}\DecValTok{35}\NormalTok{, .}\DecValTok{16}\NormalTok{, }\DecValTok{1}\NormalTok{, .}\DecValTok{59}\NormalTok{, .}\DecValTok{47}\NormalTok{)}
\NormalTok{algebra }\OtherTok{\textless{}{-}} \FunctionTok{c}\NormalTok{(.}\DecValTok{33}\NormalTok{, .}\DecValTok{32}\NormalTok{, .}\DecValTok{19}\NormalTok{, .}\DecValTok{59}\NormalTok{, }\DecValTok{1}\NormalTok{, .}\DecValTok{46}\NormalTok{)}
\NormalTok{geometry }\OtherTok{\textless{}{-}} \FunctionTok{c}\NormalTok{(.}\DecValTok{25}\NormalTok{, .}\DecValTok{33}\NormalTok{, .}\DecValTok{18}\NormalTok{, .}\DecValTok{47}\NormalTok{, .}\DecValTok{46}\NormalTok{, }\DecValTok{1}\NormalTok{)}

\NormalTok{subject.cor }\OtherTok{\textless{}{-}} \FunctionTok{cbind}\NormalTok{(french, english, history, arithmetic, algebra, geometry)}
\FunctionTok{row.names}\NormalTok{(subject.cor) }\OtherTok{\textless{}{-}} \FunctionTok{c}\NormalTok{(}\StringTok{\textquotesingle{}french\textquotesingle{}}\NormalTok{, }\StringTok{\textquotesingle{}english\textquotesingle{}}\NormalTok{, }\StringTok{\textquotesingle{}history\textquotesingle{}}\NormalTok{, }\StringTok{\textquotesingle{}arithmetic\textquotesingle{}}\NormalTok{, }\StringTok{\textquotesingle{}algebra\textquotesingle{}}\NormalTok{, }\StringTok{\textquotesingle{}geometry\textquotesingle{}}\NormalTok{)}
\FunctionTok{as.data.frame}\NormalTok{(subject.cor)}
\end{Highlighting}
\end{Shaded}

\begin{verbatim}
##            french english history arithmetic algebra geometry
## french       1.00    0.44    0.41       0.29    0.33     0.25
## english      0.44    1.00    0.35       0.35    0.32     0.33
## history      0.41    0.35    1.00       0.16    0.19     0.18
## arithmetic   0.29    0.35    0.16       1.00    0.59     0.47
## algebra      0.33    0.32    0.19       0.59    1.00     0.46
## geometry     0.25    0.33    0.18       0.47    0.46     1.00
\end{verbatim}

We have a few correlations that are unable to be compared, and will be
noted through analysis

Since we do not have a raw dataset, we assume there are no outliers.

\hypertarget{running-fa}{%
\subsubsection{Running fa}\label{running-fa}}

\begin{Shaded}
\begin{Highlighting}[]
\NormalTok{solution }\OtherTok{\textless{}{-}} \FunctionTok{fa}\NormalTok{(}\AttributeTok{r =}\NormalTok{ subject.cor, }\AttributeTok{nfactors =} \DecValTok{2}\NormalTok{, }\AttributeTok{rotate =} \StringTok{"oblimin"}\NormalTok{, }\AttributeTok{fm=}\StringTok{"pa"}\NormalTok{)}
\end{Highlighting}
\end{Shaded}

\begin{verbatim}
## Loading required namespace: GPArotation
\end{verbatim}

\begin{verbatim}
## Warning in fac(r = r, nfactors = nfactors, n.obs = n.obs, rotate = rotate, : I
## am sorry, to do these rotations requires the GPArotation package to be installed
\end{verbatim}

\begin{Shaded}
\begin{Highlighting}[]
\NormalTok{solution}
\end{Highlighting}
\end{Shaded}

\begin{verbatim}
## Factor Analysis using method =  pa
## Call: fa(r = subject.cor, nfactors = 2, rotate = "oblimin", fm = "pa")
## Standardized loadings (pattern matrix) based upon correlation matrix
##             PA1   PA2   h2   u2 com
## french     0.59  0.37 0.49 0.51 1.7
## english    0.59  0.23 0.41 0.59 1.3
## history    0.43  0.41 0.36 0.64 2.0
## arithmetic 0.71 -0.34 0.62 0.38 1.4
## algebra    0.70 -0.27 0.56 0.44 1.3
## geometry   0.58 -0.18 0.38 0.62 1.2
## 
##                        PA1  PA2
## SS loadings           2.22 0.59
## Proportion Var        0.37 0.10
## Cumulative Var        0.37 0.47
## Proportion Explained  0.79 0.21
## Cumulative Proportion 0.79 1.00
## 
## Mean item complexity =  1.5
## Test of the hypothesis that 2 factors are sufficient.
## 
## The degrees of freedom for the null model are  15  and the objective function was  1.43
## The degrees of freedom for the model are 4  and the objective function was  0.01 
## 
## The root mean square of the residuals (RMSR) is  0.01 
## The df corrected root mean square of the residuals is  0.03 
## 
## Fit based upon off diagonal values = 1
## Measures of factor score adequacy             
##                                                    PA1  PA2
## Correlation of (regression) scores with factors   0.90 0.73
## Multiple R square of scores with factors          0.82 0.53
## Minimum correlation of possible factor scores     0.63 0.06
\end{verbatim}

Overall, our model does a great job explaining \textasciitilde90\% of
variation when using 2 factors. Our most-ideal values to model from
would be arithmetic and algebra. We can also see in our console output
that hypothesis tests with 2 factors are sufficient. Neither of these
have correlations below .30.

\hypertarget{problem-3}{%
\subsection{Problem 3}\label{problem-3}}

Get data

\begin{Shaded}
\begin{Highlighting}[]
\NormalTok{food.stuff }\OtherTok{\textless{}{-}} \FunctionTok{read.table}\NormalTok{(}\StringTok{"C:/mattgraham93.github.io/school/22\_3\_DA410/data/foodstuff.dat"}\NormalTok{, }\AttributeTok{header=}\ConstantTok{TRUE}\NormalTok{)}
\NormalTok{food.stuff }\OtherTok{\textless{}{-}}\NormalTok{ food.stuff[}\SpecialCharTok{{-}}\DecValTok{1}\NormalTok{]}
\NormalTok{food.stuff}
\end{Highlighting}
\end{Shaded}

\begin{verbatim}
##    Energy Protein Fat Calcium Iron
## 1     340      20  28       9  2.6
## 2     245      21  17       9  2.7
## 3     420      15  39       7  2.0
## 4     375      19  32       9  2.5
## 5     180      22  10      17  3.7
## 6     115      20   3       8  1.4
## 7     170      25   7      12  1.5
## 8     160      26   5      14  5.9
## 9     265      20  20       9  2.6
## 10    300      18  25       9  2.3
## 11    340      20  28       9  2.5
## 12    340      19  29       9  2.5
## 13    355      19  30       9  2.4
## 14    205      18  14       7  2.5
## 15    185      23   9       9  2.7
## 16    135      22   4      25  0.6
## 17     70      11   1      82  6.0
## 18     45       7   1      74  5.4
## 19     90      14   2      38  0.8
## 20    135      16   5      15  0.5
## 21    200      19  13       5  1.0
## 22    155      16   9     157  1.8
## 23    195      16  11      14  1.3
## 24    120      17   5     159  0.7
## 25    180      22   9     367  2.5
## 26    170      25   7       7  1.2
## 27    110      23   1      98  2.6
\end{verbatim}

\hypertarget{a.-determine-factors-to-use}{%
\subsubsection{a. Determine factors to
use}\label{a.-determine-factors-to-use}}

\begin{Shaded}
\begin{Highlighting}[]
\FunctionTok{library}\NormalTok{(factoextra)}
\end{Highlighting}
\end{Shaded}

\begin{verbatim}
## Warning: package 'factoextra' was built under R version 4.2.2
\end{verbatim}

\begin{verbatim}
## Welcome! Want to learn more? See two factoextra-related books at https://goo.gl/ve3WBa
\end{verbatim}

\begin{Shaded}
\begin{Highlighting}[]
\NormalTok{food.stuff.pca }\OtherTok{\textless{}{-}} \FunctionTok{prcomp}\NormalTok{(food.stuff)}
\FunctionTok{fviz\_eig}\NormalTok{(food.stuff.pca)}
\end{Highlighting}
\end{Shaded}

\includegraphics{DA410_Exam2_MattGraham_files/figure-latex/unnamed-chunk-9-1.pdf}

We will ultimately use 2 dimensions when concluding our analysis.

\hypertarget{b---obtaining-loadings}{%
\subsubsection{B - obtaining loadings}\label{b---obtaining-loadings}}

\begin{Shaded}
\begin{Highlighting}[]
\NormalTok{S }\OtherTok{\textless{}{-}} \FunctionTok{cov}\NormalTok{(food.stuff)}
\NormalTok{R }\OtherTok{\textless{}{-}} \FunctionTok{cor}\NormalTok{(food.stuff)}
\end{Highlighting}
\end{Shaded}

S

\begin{Shaded}
\begin{Highlighting}[]
\FunctionTok{as.data.frame}\NormalTok{(S)}
\end{Highlighting}
\end{Shaded}

\begin{verbatim}
##              Energy    Protein          Fat      Calcium        Iron
## Energy  10243.01994  74.807692 1124.5655271 -2530.292023 -14.7521368
## Protein    74.80769  18.076923    1.1923077   -28.230769  -1.0846154
## Fat      1124.56553   1.192308  126.7207977  -270.673789  -0.9965812
## Calcium -2530.29202 -28.230769 -270.6737892  6089.344729   5.0491453
## Iron      -14.75214  -1.084615   -0.9965812     5.049145   2.1341026
\end{verbatim}

R

\begin{Shaded}
\begin{Highlighting}[]
\FunctionTok{as.data.frame}\NormalTok{(R)}
\end{Highlighting}
\end{Shaded}

\begin{verbatim}
##              Energy     Protein         Fat     Calcium        Iron
## Energy   1.00000000  0.17384812  0.98706740 -0.32038440 -0.09977765
## Protein  0.17384812  1.00000000  0.02491163 -0.08508934 -0.17462478
## Fat      0.98706740  0.02491163  1.00000000 -0.30813212 -0.06060118
## Calcium -0.32038440 -0.08508934 -0.30813212  1.00000000  0.04429196
## Iron    -0.09977765 -0.17462478 -0.06060118  0.04429196  1.00000000
\end{verbatim}

Get eigenvalues and eigenvectors of S and R

\begin{Shaded}
\begin{Highlighting}[]
\NormalTok{eig.S }\OtherTok{\textless{}{-}} \FunctionTok{eigen}\NormalTok{(S)}
\NormalTok{eig.R }\OtherTok{\textless{}{-}} \FunctionTok{eigen}\NormalTok{(R)}
\end{Highlighting}
\end{Shaded}

Eigen S

\begin{Shaded}
\begin{Highlighting}[]
\NormalTok{eig.S}
\end{Highlighting}
\end{Shaded}

\begin{verbatim}
## eigen() decomposition
## $values
## [1] 1.155253e+04 4.903923e+03 2.042503e+01 2.066907e+00 3.516836e-01
## 
## $vectors
##              [,1]          [,2]         [,3]          [,4]          [,5]
## [1,]  0.901061141  0.4195897978 -0.034918237 -0.0089992248  0.1035999595
## [2,]  0.006887716  0.0011983568 -0.924379029  0.1023111641 -0.3674329322
## [3,]  0.098689332  0.0474125325  0.374781853  0.0877224332 -0.9164364709
## [4,] -0.422255831  0.9064738840 -0.002194532 -0.0001874493  0.0005097596
## [5,] -0.001344580 -0.0003389534  0.061950579  0.9908361011  0.1200167645
\end{verbatim}

Eigen R

\begin{Shaded}
\begin{Highlighting}[]
\NormalTok{eig.R}
\end{Highlighting}
\end{Shaded}

\begin{verbatim}
## eigen() decomposition
## $values
## [1] 2.197777619 1.144204758 0.848574671 0.807842783 0.001600169
## 
## $vectors
##            [,1]        [,2]       [,3]      [,4]         [,5]
## [1,] -0.6539155  0.08725829 -0.1490040 0.1985936  0.709322816
## [2,] -0.1511882 -0.69052953  0.4629211 0.5245825 -0.104059181
## [3,] -0.6394332  0.20196122 -0.2157528 0.1336768 -0.697078234
## [4,]  0.3546581 -0.00633049 -0.6521357 0.6699900  0.003161132
## [5,]  0.1219811  0.68900403  0.5400663 0.4675657  0.010235855
\end{verbatim}

\hypertarget{c.-obtain-scores}{%
\subsubsection{c.~Obtain scores}\label{c.-obtain-scores}}

\begin{Shaded}
\begin{Highlighting}[]
\ControlFlowTok{for}\NormalTok{ (r }\ControlFlowTok{in}\NormalTok{ eig.R}\SpecialCharTok{$}\NormalTok{values) \{}
  \FunctionTok{print}\NormalTok{(r}\SpecialCharTok{/}\FunctionTok{sum}\NormalTok{(eig.R}\SpecialCharTok{$}\NormalTok{values))}
\NormalTok{\}}
\end{Highlighting}
\end{Shaded}

\begin{verbatim}
## [1] 0.4395555
## [1] 0.228841
## [1] 0.1697149
## [1] 0.1615686
## [1] 0.0003200338
\end{verbatim}

We can see that of our eigen values, \textasciitilde65\% of our variance
is explained with just two dimensions, and inerestingly enough going
with all 5 shows almost no meaningful value. We can see this below.

\hypertarget{plot-1}{%
\paragraph{Plot}\label{plot-1}}

\begin{Shaded}
\begin{Highlighting}[]
\FunctionTok{plot}\NormalTok{(eig.R}\SpecialCharTok{$}\NormalTok{values}\SpecialCharTok{/}\FunctionTok{sum}\NormalTok{(eig.R}\SpecialCharTok{$}\NormalTok{values), }\AttributeTok{xlab =} \StringTok{\textquotesingle{}Number of components\textquotesingle{}}\NormalTok{, }\AttributeTok{ylab=}\StringTok{\textquotesingle{}Eigen size\textquotesingle{}}\NormalTok{, }\AttributeTok{main=}\StringTok{\textquotesingle{}Plot of dimension variance\textquotesingle{}}\NormalTok{)}
\end{Highlighting}
\end{Shaded}

\includegraphics{DA410_Exam2_MattGraham_files/figure-latex/unnamed-chunk-17-1.pdf}

\hypertarget{e.-interpretation}{%
\subsubsection{e. Interpretation}\label{e.-interpretation}}

Over the impact of foods' macros pertaining to total energy, as we model
our data, we can conclude that most of our variation happens within the
first 2 measures compared to subsequent ones. This makes sense as
protein and fat are our primary determinate for overall macro tracking
and impact caloric intake.

\hypertarget{problem-4}{%
\subsection{Problem 4}\label{problem-4}}

\begin{Shaded}
\begin{Highlighting}[]
\NormalTok{scores }\OtherTok{\textless{}{-}} \FunctionTok{read.table}\NormalTok{(}\StringTok{"C:/mattgraham93.github.io/school/22\_3\_DA410/data/test\_score.dat"}\NormalTok{, }\AttributeTok{header=}\ConstantTok{TRUE}\NormalTok{)}
\NormalTok{scores }\OtherTok{\textless{}{-}}\NormalTok{ scores[}\SpecialCharTok{{-}}\DecValTok{1}\NormalTok{]}
\NormalTok{scores}
\end{Highlighting}
\end{Shaded}

\begin{verbatim}
##      math reading  sex
## 1   83.16   79.67  boy
## 2  102.51  101.13  boy
## 3   81.63   80.53  boy
## 4   88.25   84.58  boy
## 5   81.47   76.52  boy
## 6   87.19   84.70  boy
## 7   88.66   85.86  boy
## 8   79.35   81.03  boy
## 9   83.35   80.44  boy
## 10  86.58   84.67  boy
## 11  81.73   80.71  boy
## 12  85.00   81.32  boy
## 13  85.23   81.31  boy
## 14  80.30   79.37  boy
## 15  81.18   79.65  boy
## 16  88.41   85.85  boy
## 17  90.80   88.81  boy
## 18  81.68   79.71  boy
## 19  82.22   79.81  boy
## 20  78.21   74.20  boy
## 21  72.64   69.13  boy
## 22  84.61   83.05  boy
## 23  82.06   82.12  boy
## 24  87.01   84.62  boy
## 25  86.25   85.45  boy
## 26  77.05   74.03  boy
## 27  90.76   87.13  boy
## 28  81.39   78.53  boy
## 29  81.20   79.73  boy
## 30  83.07   79.94  boy
## 31  76.99   75.74 girl
## 32  83.32   81.40 girl
## 33  75.37   78.26 girl
## 34  84.81   83.93 girl
## 35  81.61   79.58 girl
## 36  76.08   78.18 girl
## 37  84.43   86.48 girl
## 38  82.29   83.16 girl
## 39  81.91   81.88 girl
## 40  97.85   97.01 girl
## 41  75.96   75.72 girl
## 42  82.47   80.84 girl
## 43  78.43   75.01 girl
## 44  82.89   82.92 girl
## 45  86.26   84.84 girl
## 46  88.48   87.90 girl
## 47  82.47   82.29 girl
## 48  87.24   87.45 girl
## 49  79.72   79.75 girl
## 50  87.52   85.44 girl
## 51  84.73   82.24 girl
## 52  77.15   77.63 girl
## 53  85.33   81.96 girl
## 54  80.58   81.67 girl
## 55  88.70   87.57 girl
## 56  79.20   77.14 girl
## 57  91.84   91.55 girl
## 58  81.07   77.01 girl
## 59  88.15   88.16 girl
## 60  76.98   75.65 girl
## 61  79.27   80.33 girl
## 62  85.70   84.27 girl
\end{verbatim}

\hypertarget{hotellings-test}{%
\subsubsection{Hotelling's test}\label{hotellings-test}}

\begin{Shaded}
\begin{Highlighting}[]
\FunctionTok{summary}\NormalTok{(}\FunctionTok{manova}\NormalTok{(}\FunctionTok{cbind}\NormalTok{(math, reading) }\SpecialCharTok{\textasciitilde{}}\NormalTok{sex, }\AttributeTok{data=}\NormalTok{scores), }\AttributeTok{test=}\StringTok{"Hotelling"}\NormalTok{)}
\end{Highlighting}
\end{Shaded}

\begin{verbatim}
##           Df Hotelling-Lawley approx F num Df den Df    Pr(>F)    
## sex        1          0.30593   9.0249      2     59 0.0003805 ***
## Residuals 60                                                      
## ---
## Signif. codes:  0 '***' 0.001 '**' 0.01 '*' 0.05 '.' 0.1 ' ' 1
\end{verbatim}

\hypertarget{hotellings-analysis}{%
\paragraph{Hotelling's Analysis}\label{hotellings-analysis}}

At alpha = 0.05 and p-value \textless{} 0.05, we can conclude there is
sufficient evidence to state there are differences between mean math and
reading scores between the recorded sexes.

\hypertarget{problem-5}{%
\subsection{Problem 5}\label{problem-5}}

Get data

\begin{Shaded}
\begin{Highlighting}[]
\NormalTok{glucose }\OtherTok{\textless{}{-}} \FunctionTok{read.table}\NormalTok{(}\StringTok{"C:/mattgraham93.github.io/school/22\_3\_DA410/data/T3\_5\_DIABETES.DAT"}\NormalTok{, }\AttributeTok{header=}\ConstantTok{FALSE}\NormalTok{)[}\DecValTok{1}\SpecialCharTok{:}\DecValTok{34}\NormalTok{,]}
\NormalTok{glucose }\OtherTok{\textless{}{-}}\NormalTok{ glucose[}\SpecialCharTok{{-}}\DecValTok{1}\NormalTok{]}
\FunctionTok{colnames}\NormalTok{(glucose) }\OtherTok{\textless{}{-}} \FunctionTok{c}\NormalTok{(}\StringTok{\textquotesingle{}rel\_wt\textquotesingle{}}\NormalTok{, }\StringTok{\textquotesingle{}fst\_pls\_glu\textquotesingle{}}\NormalTok{, }\StringTok{\textquotesingle{}gl\_int\textquotesingle{}}\NormalTok{, }\StringTok{\textquotesingle{}ins\_resp\textquotesingle{}}\NormalTok{, }\StringTok{\textquotesingle{}ins\_resist\textquotesingle{}}\NormalTok{)  }

\NormalTok{ys }\OtherTok{\textless{}{-}}\NormalTok{ glucose[}\DecValTok{1}\SpecialCharTok{:}\DecValTok{2}\NormalTok{]}
\NormalTok{xs }\OtherTok{\textless{}{-}}\NormalTok{ glucose[}\DecValTok{3}\SpecialCharTok{:}\DecValTok{5}\NormalTok{]}

\NormalTok{glucose}
\end{Highlighting}
\end{Shaded}

\begin{verbatim}
##    rel_wt fst_pls_glu gl_int ins_resp ins_resist
## 1    0.81          80    356      124         55
## 2    0.95          97    289      117         76
## 3    0.94         105    319      143        105
## 4    1.04          90    356      199        108
## 5    1.00          90    323      240        143
## 6    0.76          86    381      157        165
## 7    0.91         100    350      221        119
## 8    1.10          85    301      186        105
## 9    0.99          97    379      142         98
## 10   0.78          97    296      131         94
## 11   0.90          91    353      221         53
## 12   0.73          87    306      178         66
## 13   0.96          78    290      136        142
## 14   0.84          90    371      200         93
## 15   0.74          86    312      208         68
## 16   0.98          80    393      202        102
## 17   1.10          90    364      152         76
## 18   0.85          99    359      185         37
## 19   0.83          85    296      116         60
## 20   0.93          90    345      123         50
## 21   0.95          90    378      136         47
## 22   0.74          88    304      134         50
## 23   0.95          95    347      184         91
## 24   0.97          90    327      192        124
## 25   0.72          92    386      279         74
## 26   1.11          74    365      228        235
## 27   1.20          98    365      145        158
## 28   1.13         100    352      172        140
## 29   1.00          86    325      179        145
## 30   0.78          98    321      222         99
## 31   1.00          70    360      134         90
## 32   1.00          99    336      143        105
## 33   0.71          75    352      169         32
## 34   0.76          90    353      263        165
\end{verbatim}

Find means

\begin{Shaded}
\begin{Highlighting}[]
\NormalTok{x.bar }\OtherTok{\textless{}{-}} \FunctionTok{colMeans}\NormalTok{(xs)}
\NormalTok{x.bar}
\end{Highlighting}
\end{Shaded}

\begin{verbatim}
##     gl_int   ins_resp ins_resist 
##  341.47059  175.32353   99.11765
\end{verbatim}

\begin{Shaded}
\begin{Highlighting}[]
\NormalTok{y.bar }\OtherTok{\textless{}{-}} \FunctionTok{colMeans}\NormalTok{(ys)}
\NormalTok{y.bar}
\end{Highlighting}
\end{Shaded}

\begin{verbatim}
##      rel_wt fst_pls_glu 
##   0.9164706  89.6470588
\end{verbatim}

\hypertarget{a---find-canonical-correlations}{%
\subsubsection{a - Find canonical
correlations}\label{a---find-canonical-correlations}}

\begin{Shaded}
\begin{Highlighting}[]
\NormalTok{cancor2}\OtherTok{\textless{}{-}}\ControlFlowTok{function}\NormalTok{(x,y,}\AttributeTok{dec=}\DecValTok{4}\NormalTok{)\{ }
\CommentTok{\#Canonical Correlation Analysis to mimic SAS PROC CANCOR output. }
\CommentTok{\#Basic formulas can be found in Chapter 10 of Mardia, Kent, and Bibby (1979). }
\CommentTok{\# The approximate F statistic is exercise 3.7.6b. }
\NormalTok{    x}\OtherTok{\textless{}{-}}\FunctionTok{as.matrix}\NormalTok{(x)}
\NormalTok{    y}\OtherTok{\textless{}{-}}\FunctionTok{as.matrix}\NormalTok{(y) }
    
\NormalTok{    n}\OtherTok{\textless{}{-}}\FunctionTok{dim}\NormalTok{(x)[}\DecValTok{1}\NormalTok{]}
\NormalTok{    q1}\OtherTok{\textless{}{-}}\FunctionTok{dim}\NormalTok{(x)[}\DecValTok{2}\NormalTok{]}
\NormalTok{    q2}\OtherTok{\textless{}{-}}\FunctionTok{dim}\NormalTok{(y)[}\DecValTok{2}\NormalTok{]}
\NormalTok{    q}\OtherTok{\textless{}{-}}\FunctionTok{min}\NormalTok{(q1,q2) }
    
\NormalTok{    S11}\OtherTok{\textless{}{-}}\FunctionTok{cov}\NormalTok{(x)}
\NormalTok{    S12}\OtherTok{\textless{}{-}}\FunctionTok{cov}\NormalTok{(x,y)}
\NormalTok{    S21}\OtherTok{\textless{}{-}}\FunctionTok{t}\NormalTok{(S12)}
\NormalTok{    S22}\OtherTok{\textless{}{-}}\FunctionTok{cov}\NormalTok{(y) }
    
\NormalTok{    E1}\OtherTok{\textless{}{-}}\FunctionTok{eigen}\NormalTok{(}\FunctionTok{solve}\NormalTok{(S11)}\SpecialCharTok{\%*\%}\NormalTok{S12}\SpecialCharTok{\%*\%}\FunctionTok{solve}\NormalTok{(S22)}\SpecialCharTok{\%*\%}\NormalTok{S21)}
\NormalTok{    E2}\OtherTok{\textless{}{-}}\FunctionTok{eigen}\NormalTok{(}\FunctionTok{solve}\NormalTok{(S22)}\SpecialCharTok{\%*\%}\NormalTok{S21}\SpecialCharTok{\%*\%}\FunctionTok{solve}\NormalTok{(S11)}\SpecialCharTok{\%*\%}\NormalTok{S12) }
    
\NormalTok{    rsquared}\OtherTok{\textless{}{-}}\FunctionTok{as.double}\NormalTok{(E1}\SpecialCharTok{$}\NormalTok{values[}\DecValTok{1}\SpecialCharTok{:}\NormalTok{q]) }
    
\NormalTok{    LR}\OtherTok{\textless{}{-}}\ConstantTok{NULL}\NormalTok{;pp}\OtherTok{\textless{}{-}}\ConstantTok{NULL}\NormalTok{;qq}\OtherTok{\textless{}{-}}\ConstantTok{NULL}\NormalTok{;tt}\OtherTok{\textless{}{-}}\ConstantTok{NULL} 
    
    \ControlFlowTok{for}\NormalTok{ (i }\ControlFlowTok{in} \DecValTok{1}\SpecialCharTok{:}\NormalTok{q)\{ }
\NormalTok{        LR}\OtherTok{\textless{}{-}}\FunctionTok{c}\NormalTok{(LR,}\FunctionTok{prod}\NormalTok{(}\DecValTok{1}\SpecialCharTok{{-}}\NormalTok{rsquared[i}\SpecialCharTok{:}\NormalTok{q])) }
\NormalTok{        pp}\OtherTok{\textless{}{-}}\FunctionTok{c}\NormalTok{(pp,q1}\SpecialCharTok{{-}}\NormalTok{i}\SpecialCharTok{+}\DecValTok{1}\NormalTok{) }
\NormalTok{        qq}\OtherTok{\textless{}{-}}\FunctionTok{c}\NormalTok{(qq,q2}\SpecialCharTok{{-}}\NormalTok{i}\SpecialCharTok{+}\DecValTok{1}\NormalTok{) }
\NormalTok{        tt}\OtherTok{\textless{}{-}}\FunctionTok{c}\NormalTok{(tt,n}\DecValTok{{-}1}\SpecialCharTok{{-}}\NormalTok{i}\SpecialCharTok{+}\DecValTok{1}\NormalTok{)\} }
    
\NormalTok{    m}\OtherTok{\textless{}{-}}\NormalTok{tt}\FloatTok{{-}0.5}\SpecialCharTok{*}\NormalTok{(pp}\SpecialCharTok{+}\NormalTok{qq}\SpecialCharTok{+}\DecValTok{1}\NormalTok{);lambda}\OtherTok{\textless{}{-}}\NormalTok{(}\DecValTok{1}\SpecialCharTok{/}\DecValTok{4}\NormalTok{)}\SpecialCharTok{*}\NormalTok{(pp}\SpecialCharTok{*}\NormalTok{qq}\DecValTok{{-}2}\NormalTok{);s}\OtherTok{\textless{}{-}}\FunctionTok{sqrt}\NormalTok{((pp}\SpecialCharTok{\^{}}\DecValTok{2}\SpecialCharTok{*}\NormalTok{qq}\SpecialCharTok{\^{}}\DecValTok{2{-}4}\NormalTok{)}\SpecialCharTok{/}\NormalTok{(pp}\SpecialCharTok{\^{}}\DecValTok{2}\SpecialCharTok{+}\NormalTok{qq}\SpecialCharTok{\^{}}\DecValTok{2{-}5}\NormalTok{)) }
\NormalTok{    F}\OtherTok{\textless{}{-}}\NormalTok{((m}\SpecialCharTok{*}\NormalTok{s}\DecValTok{{-}2}\SpecialCharTok{*}\NormalTok{lambda)}\SpecialCharTok{/}\NormalTok{(pp}\SpecialCharTok{*}\NormalTok{qq))}\SpecialCharTok{*}\NormalTok{((}\DecValTok{1}\SpecialCharTok{{-}}\NormalTok{LR}\SpecialCharTok{\^{}}\NormalTok{(}\DecValTok{1}\SpecialCharTok{/}\NormalTok{s))}\SpecialCharTok{/}\NormalTok{LR}\SpecialCharTok{\^{}}\NormalTok{(}\DecValTok{1}\SpecialCharTok{/}\NormalTok{s))}
\NormalTok{    df1}\OtherTok{\textless{}{-}}\NormalTok{pp}\SpecialCharTok{*}\NormalTok{qq;df2}\OtherTok{\textless{}{-}}\NormalTok{(m}\SpecialCharTok{*}\NormalTok{s}\DecValTok{{-}2}\SpecialCharTok{*}\NormalTok{lambda)}
\NormalTok{    pval}\OtherTok{\textless{}{-}}\DecValTok{1}\SpecialCharTok{{-}}\FunctionTok{pf}\NormalTok{(F,df1,df2) }
\NormalTok{    outmat}\OtherTok{\textless{}{-}}\FunctionTok{round}\NormalTok{(}\FunctionTok{cbind}\NormalTok{(}\FunctionTok{sqrt}\NormalTok{(rsquared),rsquared,LR,F,df1,df2,pval),dec) }
      
    \FunctionTok{colnames}\NormalTok{(outmat) }\OtherTok{\textless{}{-}} \FunctionTok{list}\NormalTok{(}\StringTok{"R"}\NormalTok{,}\StringTok{"RSquared"}\NormalTok{,}\StringTok{"LR"}\NormalTok{,}\StringTok{"ApproxF"}\NormalTok{,}\StringTok{"NumDF"}\NormalTok{,}\StringTok{"DenDF"}\NormalTok{,}\StringTok{"pvalue"}\NormalTok{)}
    \FunctionTok{rownames}\NormalTok{(outmat) }\OtherTok{\textless{}{-}} \FunctionTok{as.character}\NormalTok{(}\DecValTok{1}\SpecialCharTok{:}\NormalTok{q)}
\NormalTok{    xrels}\OtherTok{\textless{}{-}}\FunctionTok{round}\NormalTok{(}\FunctionTok{cor}\NormalTok{(x,x}\SpecialCharTok{\%*\%}\NormalTok{E1}\SpecialCharTok{$}\NormalTok{vectors)[,}\DecValTok{1}\SpecialCharTok{:}\NormalTok{q],dec) }
    \FunctionTok{colnames}\NormalTok{(xrels)}\OtherTok{\textless{}{-}}\FunctionTok{apply}\NormalTok{(}\FunctionTok{cbind}\NormalTok{(}\FunctionTok{rep}\NormalTok{(}\StringTok{"U"}\NormalTok{,q),}\FunctionTok{as.character}\NormalTok{(}\DecValTok{1}\SpecialCharTok{:}\NormalTok{q)),}\DecValTok{1}\NormalTok{,paste,}\AttributeTok{collapse=}\StringTok{""}\NormalTok{)}
\NormalTok{    yrels}\OtherTok{\textless{}{-}}\FunctionTok{round}\NormalTok{(}\FunctionTok{cor}\NormalTok{(y,y}\SpecialCharTok{\%*\%}\NormalTok{E2}\SpecialCharTok{$}\NormalTok{vectors)[,}\DecValTok{1}\SpecialCharTok{:}\NormalTok{q],dec) }
    \FunctionTok{colnames}\NormalTok{(yrels)}\OtherTok{\textless{}{-}} \FunctionTok{apply}\NormalTok{(}\FunctionTok{cbind}\NormalTok{(}\FunctionTok{rep}\NormalTok{(}\StringTok{"V"}\NormalTok{,q),}\FunctionTok{as.character}\NormalTok{(}\DecValTok{1}\SpecialCharTok{:}\NormalTok{q)),}\DecValTok{1}\NormalTok{,paste,}\AttributeTok{collapse=}\StringTok{""}\NormalTok{)}
    \FunctionTok{list}\NormalTok{(}\AttributeTok{Summary=}\NormalTok{outmat,}
         \AttributeTok{a.Coefficients=}\NormalTok{E1}\SpecialCharTok{$}\NormalTok{vectors,}
         \AttributeTok{b.Coefficients=}\NormalTok{E2}\SpecialCharTok{$}\NormalTok{vectors, }
         \AttributeTok{XUCorrelations=}\NormalTok{xrels,}\AttributeTok{YVCorrelations=}\NormalTok{yrels}
\NormalTok{     ) }
\NormalTok{   \}  }
\DocumentationTok{\#\# }\RegionMarkerTok{END}\DocumentationTok{ FUNCTION }
\DocumentationTok{\#\#\#\#\#\#\#\#\#\#\#\#\#\#\#\#\#\#\#\#\#\#\#\#\#\#\#\#\#\#\#\#\#\#\#\#\#\#\#\#\#\#\#\#\#\#\#\#\# }
\end{Highlighting}
\end{Shaded}

\hypertarget{b---find-standard-coefficients}{%
\subsubsection{b - Find standard
coefficients}\label{b---find-standard-coefficients}}

For canonical variables

Fasting coefficients

\begin{Shaded}
\begin{Highlighting}[]
\NormalTok{before.coefficients }\OtherTok{\textless{}{-}} \FunctionTok{cancor2}\NormalTok{(xs, ys)}\SpecialCharTok{$}\NormalTok{a.Coefficients}
\NormalTok{after.coefficients }\OtherTok{\textless{}{-}} \FunctionTok{cancor2}\NormalTok{(xs, ys)}\SpecialCharTok{$}\NormalTok{b.Coefficients}

\FunctionTok{diag}\NormalTok{(before.coefficients)}
\end{Highlighting}
\end{Shaded}

\begin{verbatim}
## [1] 0.42680635 0.07577154 0.27703237
\end{verbatim}

Post-consumption coefficients

\begin{Shaded}
\begin{Highlighting}[]
\FunctionTok{diag}\NormalTok{(after.coefficients)}
\end{Highlighting}
\end{Shaded}

\begin{verbatim}
## [1] 0.99999176 0.09433123
\end{verbatim}

\hypertarget{c---test-significance-for-reach-canonical-correlation}{%
\subsubsection{c - Test significance for reach canonical
correlation}\label{c---test-significance-for-reach-canonical-correlation}}

\begin{Shaded}
\begin{Highlighting}[]
\FunctionTok{cancor2}\NormalTok{(xs, ys) }
\end{Highlighting}
\end{Shaded}

\begin{verbatim}
## $Summary
##        R RSquared     LR ApproxF NumDF DenDF pvalue
## 1 0.6024   0.3628 0.6353  2.4616     6    58 0.0345
## 2 0.0546   0.0030 0.9970     NaN     2   NaN    NaN
## 
## $a.Coefficients
##            [,1]        [,2]       [,3]
## [1,]  0.4268063 -0.96431194 0.06551464
## [2,] -0.5231307  0.07577154 0.95862448
## [3,]  0.7376792  0.25369501 0.27703237
## 
## $b.Coefficients
##              [,1]       [,2]
## [1,]  0.999991761 0.99554087
## [2,] -0.004059221 0.09433123
## 
## $XUCorrelations
##                 U1      U2
## gl_int      0.2754 -0.9065
## ins_resp   -0.2448 -0.0518
## ins_resist  0.7781  0.3186
## 
## $YVCorrelations
##                  V1     V2
## rel_wt       0.9691 0.2465
## fst_pls_glu -0.1661 0.9861
\end{verbatim}

\begin{Shaded}
\begin{Highlighting}[]
\CommentTok{\# It produces two other pieces of information:  An F{-}test for the  }
\CommentTok{\# significance of each canonical correlation, and the correlations between  }
\CommentTok{\# the original variables and the corresponding canonical variates. }
\end{Highlighting}
\end{Shaded}

\hypertarget{interpretation}{%
\paragraph{Interpretation}\label{interpretation}}

Given all our p-values \textless{} 0.05, there is enough evidence to
conclude that there is at least one non-zero canonical correlation
between relative weight and plasma glucose across glucose intolerance,
insulin response to oral glucose, and insulin resistance.This means our
subjects had different responses to ingesting glucose. This makes sense
as diabetes and insulin are directly correlated.

\hypertarget{problem-6}{%
\subsection{Problem 6}\label{problem-6}}

\begin{Shaded}
\begin{Highlighting}[]
\NormalTok{hematology }\OtherTok{\textless{}{-}} \FunctionTok{read.csv}\NormalTok{(}\StringTok{\textquotesingle{}C:/mattgraham93.github.io/school/22\_3\_DA410/data/hematology.csv\textquotesingle{}}\NormalTok{, }\AttributeTok{header=}\ConstantTok{TRUE}\NormalTok{)}
\NormalTok{hematology}
\end{Highlighting}
\end{Shaded}

\begin{verbatim}
##    Observation.number X..1 X..2 X..3 X..4 X..5 X..6
## 1                   1 13.4   39 4100   14   25   17
## 2                   2 14.6   46 5000   15   30   20
## 3                   3 13.5   42 4500   19   21   18
## 4                   4 15.0   46 4600   23   16   18
## 5                   5 14.6   44 5100   17   31   19
## 6                   6 14.0   44 4900   20   24   19
## 7                   7 16.4   49 4300   21   17   18
## 8                   8 14.8   44 4400   16   26   29
## 9                   9 15.2   46 4100   27   13   27
## 10                 10 15.5   48 8400   34   42   36
## 11                 11 15.2   47 5600   26   27   22
## 12                 12 16.9   50 5100   28   17   23
## 13                 13 14.8   44 4700   24   20   23
## 14                 14 16.2   45 5600   26   25   19
## 15                 15 14.7   43 4000   23   13   17
\end{verbatim}

\begin{Shaded}
\begin{Highlighting}[]
\FunctionTok{pairs}\NormalTok{(hematology[}\SpecialCharTok{{-}}\DecValTok{1}\NormalTok{])}
\end{Highlighting}
\end{Shaded}

\includegraphics{DA410_Exam2_MattGraham_files/figure-latex/unnamed-chunk-28-1.pdf}

Normalizing

\begin{Shaded}
\begin{Highlighting}[]
\NormalTok{z }\OtherTok{\textless{}{-}}\NormalTok{ hematology[,}\SpecialCharTok{{-}}\FunctionTok{c}\NormalTok{(}\DecValTok{1}\NormalTok{,}\DecValTok{1}\NormalTok{)]}
\NormalTok{means }\OtherTok{\textless{}{-}} \FunctionTok{apply}\NormalTok{(z,}\DecValTok{2}\NormalTok{,mean)}
\NormalTok{sds }\OtherTok{\textless{}{-}} \FunctionTok{apply}\NormalTok{(z,}\DecValTok{2}\NormalTok{,sd)}
\NormalTok{nor }\OtherTok{\textless{}{-}} \FunctionTok{scale}\NormalTok{(z,}\AttributeTok{center=}\NormalTok{means,}\AttributeTok{scale=}\NormalTok{sds)}

\NormalTok{distance }\OtherTok{=} \FunctionTok{dist}\NormalTok{(nor)}
\end{Highlighting}
\end{Shaded}

Plotting

\begin{Shaded}
\begin{Highlighting}[]
\NormalTok{mydata.hclust }\OtherTok{=} \FunctionTok{hclust}\NormalTok{(distance)}
\FunctionTok{plot}\NormalTok{(mydata.hclust)}
\end{Highlighting}
\end{Shaded}

\includegraphics{DA410_Exam2_MattGraham_files/figure-latex/unnamed-chunk-30-1.pdf}

\begin{Shaded}
\begin{Highlighting}[]
\FunctionTok{plot}\NormalTok{(mydata.hclust,}\AttributeTok{labels=}\NormalTok{hematology}\SpecialCharTok{$}\NormalTok{Observation.number,}\AttributeTok{main=}\StringTok{\textquotesingle{}Default from hclust\textquotesingle{}}\NormalTok{)}
\end{Highlighting}
\end{Shaded}

\includegraphics{DA410_Exam2_MattGraham_files/figure-latex/unnamed-chunk-30-2.pdf}

\begin{Shaded}
\begin{Highlighting}[]
\FunctionTok{plot}\NormalTok{(mydata.hclust,}\AttributeTok{hang=}\SpecialCharTok{{-}}\DecValTok{1}\NormalTok{, }\AttributeTok{labels=}\NormalTok{hematology}\SpecialCharTok{$}\NormalTok{Observation.number,}\AttributeTok{main=}\StringTok{\textquotesingle{}Default from hclust\textquotesingle{}}\NormalTok{)}
\end{Highlighting}
\end{Shaded}

\includegraphics{DA410_Exam2_MattGraham_files/figure-latex/unnamed-chunk-30-3.pdf}

Average linkage

\begin{Shaded}
\begin{Highlighting}[]
\NormalTok{mydata.hclust}\OtherTok{\textless{}{-}}\FunctionTok{hclust}\NormalTok{(distance,}\AttributeTok{method=}\StringTok{"average"}\NormalTok{) }
\FunctionTok{plot}\NormalTok{(mydata.hclust,}\AttributeTok{hang=}\SpecialCharTok{{-}}\DecValTok{1}\NormalTok{) }
\end{Highlighting}
\end{Shaded}

\includegraphics{DA410_Exam2_MattGraham_files/figure-latex/unnamed-chunk-31-1.pdf}

\end{document}
