% Options for packages loaded elsewhere
\PassOptionsToPackage{unicode}{hyperref}
\PassOptionsToPackage{hyphens}{url}
%
\documentclass[
]{article}
\usepackage{amsmath,amssymb}
\usepackage{lmodern}
\usepackage{iftex}
\ifPDFTeX
  \usepackage[T1]{fontenc}
  \usepackage[utf8]{inputenc}
  \usepackage{textcomp} % provide euro and other symbols
\else % if luatex or xetex
  \usepackage{unicode-math}
  \defaultfontfeatures{Scale=MatchLowercase}
  \defaultfontfeatures[\rmfamily]{Ligatures=TeX,Scale=1}
\fi
% Use upquote if available, for straight quotes in verbatim environments
\IfFileExists{upquote.sty}{\usepackage{upquote}}{}
\IfFileExists{microtype.sty}{% use microtype if available
  \usepackage[]{microtype}
  \UseMicrotypeSet[protrusion]{basicmath} % disable protrusion for tt fonts
}{}
\makeatletter
\@ifundefined{KOMAClassName}{% if non-KOMA class
  \IfFileExists{parskip.sty}{%
    \usepackage{parskip}
  }{% else
    \setlength{\parindent}{0pt}
    \setlength{\parskip}{6pt plus 2pt minus 1pt}}
}{% if KOMA class
  \KOMAoptions{parskip=half}}
\makeatother
\usepackage{xcolor}
\usepackage[margin=1in]{geometry}
\usepackage{color}
\usepackage{fancyvrb}
\newcommand{\VerbBar}{|}
\newcommand{\VERB}{\Verb[commandchars=\\\{\}]}
\DefineVerbatimEnvironment{Highlighting}{Verbatim}{commandchars=\\\{\}}
% Add ',fontsize=\small' for more characters per line
\usepackage{framed}
\definecolor{shadecolor}{RGB}{248,248,248}
\newenvironment{Shaded}{\begin{snugshade}}{\end{snugshade}}
\newcommand{\AlertTok}[1]{\textcolor[rgb]{0.94,0.16,0.16}{#1}}
\newcommand{\AnnotationTok}[1]{\textcolor[rgb]{0.56,0.35,0.01}{\textbf{\textit{#1}}}}
\newcommand{\AttributeTok}[1]{\textcolor[rgb]{0.77,0.63,0.00}{#1}}
\newcommand{\BaseNTok}[1]{\textcolor[rgb]{0.00,0.00,0.81}{#1}}
\newcommand{\BuiltInTok}[1]{#1}
\newcommand{\CharTok}[1]{\textcolor[rgb]{0.31,0.60,0.02}{#1}}
\newcommand{\CommentTok}[1]{\textcolor[rgb]{0.56,0.35,0.01}{\textit{#1}}}
\newcommand{\CommentVarTok}[1]{\textcolor[rgb]{0.56,0.35,0.01}{\textbf{\textit{#1}}}}
\newcommand{\ConstantTok}[1]{\textcolor[rgb]{0.00,0.00,0.00}{#1}}
\newcommand{\ControlFlowTok}[1]{\textcolor[rgb]{0.13,0.29,0.53}{\textbf{#1}}}
\newcommand{\DataTypeTok}[1]{\textcolor[rgb]{0.13,0.29,0.53}{#1}}
\newcommand{\DecValTok}[1]{\textcolor[rgb]{0.00,0.00,0.81}{#1}}
\newcommand{\DocumentationTok}[1]{\textcolor[rgb]{0.56,0.35,0.01}{\textbf{\textit{#1}}}}
\newcommand{\ErrorTok}[1]{\textcolor[rgb]{0.64,0.00,0.00}{\textbf{#1}}}
\newcommand{\ExtensionTok}[1]{#1}
\newcommand{\FloatTok}[1]{\textcolor[rgb]{0.00,0.00,0.81}{#1}}
\newcommand{\FunctionTok}[1]{\textcolor[rgb]{0.00,0.00,0.00}{#1}}
\newcommand{\ImportTok}[1]{#1}
\newcommand{\InformationTok}[1]{\textcolor[rgb]{0.56,0.35,0.01}{\textbf{\textit{#1}}}}
\newcommand{\KeywordTok}[1]{\textcolor[rgb]{0.13,0.29,0.53}{\textbf{#1}}}
\newcommand{\NormalTok}[1]{#1}
\newcommand{\OperatorTok}[1]{\textcolor[rgb]{0.81,0.36,0.00}{\textbf{#1}}}
\newcommand{\OtherTok}[1]{\textcolor[rgb]{0.56,0.35,0.01}{#1}}
\newcommand{\PreprocessorTok}[1]{\textcolor[rgb]{0.56,0.35,0.01}{\textit{#1}}}
\newcommand{\RegionMarkerTok}[1]{#1}
\newcommand{\SpecialCharTok}[1]{\textcolor[rgb]{0.00,0.00,0.00}{#1}}
\newcommand{\SpecialStringTok}[1]{\textcolor[rgb]{0.31,0.60,0.02}{#1}}
\newcommand{\StringTok}[1]{\textcolor[rgb]{0.31,0.60,0.02}{#1}}
\newcommand{\VariableTok}[1]{\textcolor[rgb]{0.00,0.00,0.00}{#1}}
\newcommand{\VerbatimStringTok}[1]{\textcolor[rgb]{0.31,0.60,0.02}{#1}}
\newcommand{\WarningTok}[1]{\textcolor[rgb]{0.56,0.35,0.01}{\textbf{\textit{#1}}}}
\usepackage{graphicx}
\makeatletter
\def\maxwidth{\ifdim\Gin@nat@width>\linewidth\linewidth\else\Gin@nat@width\fi}
\def\maxheight{\ifdim\Gin@nat@height>\textheight\textheight\else\Gin@nat@height\fi}
\makeatother
% Scale images if necessary, so that they will not overflow the page
% margins by default, and it is still possible to overwrite the defaults
% using explicit options in \includegraphics[width, height, ...]{}
\setkeys{Gin}{width=\maxwidth,height=\maxheight,keepaspectratio}
% Set default figure placement to htbp
\makeatletter
\def\fps@figure{htbp}
\makeatother
\setlength{\emergencystretch}{3em} % prevent overfull lines
\providecommand{\tightlist}{%
  \setlength{\itemsep}{0pt}\setlength{\parskip}{0pt}}
\setcounter{secnumdepth}{-\maxdimen} % remove section numbering
\ifLuaTeX
  \usepackage{selnolig}  % disable illegal ligatures
\fi
\IfFileExists{bookmark.sty}{\usepackage{bookmark}}{\usepackage{hyperref}}
\IfFileExists{xurl.sty}{\usepackage{xurl}}{} % add URL line breaks if available
\urlstyle{same} % disable monospaced font for URLs
\hypersetup{
  pdftitle={DA410\_Project1\_MattGraham},
  hidelinks,
  pdfcreator={LaTeX via pandoc}}

\title{DA410\_Project1\_MattGraham}
\author{}
\date{\vspace{-2.5em}}

\begin{document}
\maketitle

This is our first project, analyzing air pollution, mortality rates, and
relevant parameters.

\hypertarget{getting-imports-for-analysis}{%
\paragraph{Getting imports for
analysis}\label{getting-imports-for-analysis}}

\begin{Shaded}
\begin{Highlighting}[]
\FunctionTok{library}\NormalTok{(nnspat)  }\CommentTok{\# used for dist2full()}
\FunctionTok{library}\NormalTok{(}\StringTok{"dplyr"}\NormalTok{)  }\CommentTok{\# used to select numeric datatypes}
\FunctionTok{library}\NormalTok{(}\StringTok{"ggplot2"}\NormalTok{)}
\FunctionTok{library}\NormalTok{(reshape)  }\CommentTok{\# used for melting matricies}

\NormalTok{airpol.full }\OtherTok{\textless{}{-}} \FunctionTok{read.table}\NormalTok{(}\StringTok{"C:/mattgraham93.github.io/school/22\_3\_DA410/data/airpoll.txt"}\NormalTok{, }\AttributeTok{header=}\ConstantTok{TRUE}\NormalTok{)}

\NormalTok{city.names }\OtherTok{\textless{}{-}} \FunctionTok{as.character}\NormalTok{(airpol.full[}\DecValTok{1}\SpecialCharTok{:}\DecValTok{16}\NormalTok{,}\DecValTok{1}\NormalTok{])}
\NormalTok{airpol.data.sub }\OtherTok{\textless{}{-}}\NormalTok{ airpol.full[}\DecValTok{1}\SpecialCharTok{:}\DecValTok{16}\NormalTok{,}\DecValTok{2}\SpecialCharTok{:}\DecValTok{8}\NormalTok{]}

\NormalTok{airpol.data.sub}
\end{Highlighting}
\end{Shaded}

\begin{verbatim}
##    Rainfall Education Popden Nonwhite NOX SO2 Mortality
## 1        36      11.4   3243      8.8  15  59     921.9
## 2        35      11.0   4281      3.5  10  39     997.9
## 3        44       9.8   4260      0.8   6  33     962.4
## 4        47      11.1   3125     27.1   8  24     982.3
## 5        43       9.6   6441     24.4  38 206    1071.0
## 6        53      10.2   3325     38.5  32  72    1030.0
## 7        43      12.1   4679      3.5  32  62     934.7
## 8        45      10.6   2140      5.3   4   4     899.5
## 9        36      10.5   6582      8.1  12  37    1002.0
## 10       36      10.7   4213      6.7   7  20     912.3
## 11       52       9.6   2302     22.2   8  27    1018.0
## 12       33      10.9   6122     16.3  63 278    1025.0
## 13       40      10.2   4101     13.0  26 146     970.5
## 14       35      11.1   3042     14.7  21  64     986.0
## 15       37      11.9   4259     13.1   9  15     958.8
## 16       35      11.8   1441     14.8   1   1     860.1
\end{verbatim}

\hypertarget{calculating-covariance-and-correlation}{%
\subsubsection{1 - Calculating covariance and
correlation}\label{calculating-covariance-and-correlation}}

Identify which pairs of variables seem to be strongly associated, and
describe the nature (strength and direction) of the relationship between
these variable pairs.

\hypertarget{covariance}{%
\paragraph{Covariance}\label{covariance}}

\begin{Shaded}
\begin{Highlighting}[]
\NormalTok{my.S }\OtherTok{\textless{}{-}} \FunctionTok{round}\NormalTok{(}\FunctionTok{var}\NormalTok{(airpol.data.sub), }\AttributeTok{digits=}\DecValTok{2}\NormalTok{)}

\NormalTok{melted.my.S }\OtherTok{\textless{}{-}} \FunctionTok{melt}\NormalTok{(my.S)}

\FunctionTok{ggplot}\NormalTok{(}\AttributeTok{data =}\NormalTok{ melted.my.S, }\FunctionTok{aes}\NormalTok{(}\AttributeTok{x=}\NormalTok{X1, }\AttributeTok{y=}\NormalTok{X2, }\AttributeTok{fill=}\NormalTok{value)) }\SpecialCharTok{+} 
  \FunctionTok{geom\_tile}\NormalTok{() }\SpecialCharTok{+}
  \FunctionTok{geom\_text}\NormalTok{(}\FunctionTok{aes}\NormalTok{(}\AttributeTok{label =} \FunctionTok{round}\NormalTok{(value, }\DecValTok{1}\NormalTok{)))}
\end{Highlighting}
\end{Shaded}

\includegraphics{Project1_covariance_correlation_distance_plots_files/figure-latex/unnamed-chunk-1-1.pdf}

\hypertarget{correlation}{%
\paragraph{Correlation}\label{correlation}}

\begin{Shaded}
\begin{Highlighting}[]
\NormalTok{my.R }\OtherTok{\textless{}{-}} \FunctionTok{round}\NormalTok{(}\FunctionTok{cor}\NormalTok{(airpol.data.sub), }\AttributeTok{digits=}\DecValTok{2}\NormalTok{)}

\NormalTok{melted.my.R }\OtherTok{\textless{}{-}} \FunctionTok{melt}\NormalTok{(my.R)}

\FunctionTok{ggplot}\NormalTok{(}\AttributeTok{data =}\NormalTok{ melted.my.R, }\FunctionTok{aes}\NormalTok{(}\AttributeTok{x=}\NormalTok{X1, }\AttributeTok{y=}\NormalTok{X2, }\AttributeTok{fill=}\NormalTok{value)) }\SpecialCharTok{+} 
  \FunctionTok{geom\_tile}\NormalTok{() }\SpecialCharTok{+}
  \FunctionTok{geom\_text}\NormalTok{(}\FunctionTok{aes}\NormalTok{(}\AttributeTok{label =} \FunctionTok{round}\NormalTok{(value, }\DecValTok{1}\NormalTok{)))}
\end{Highlighting}
\end{Shaded}

\includegraphics{Project1_covariance_correlation_distance_plots_files/figure-latex/unnamed-chunk-2-1.pdf}

\hypertarget{covariance-vs.-correlation}{%
\paragraph{Covariance
vs.~Correlation}\label{covariance-vs.-correlation}}

\begin{Shaded}
\begin{Highlighting}[]
\FunctionTok{as.data.frame}\NormalTok{(}\FunctionTok{cov2cor}\NormalTok{(my.S))}
\end{Highlighting}
\end{Shaded}

\begin{verbatim}
##             Rainfall   Education     Popden   Nonwhite         NOX        SO2
## Rainfall   1.0000000 -0.49976469 -0.2939816  0.5419172 -0.08035640 -0.1737525
## Education -0.4997647  1.00000000 -0.1924363 -0.2842847 -0.08906087 -0.2663758
## Popden    -0.2939816 -0.19243630  1.0000000 -0.1100481  0.58408307  0.6185629
## Nonwhite   0.5419172 -0.28428472 -0.1100481  1.0000000  0.31286146  0.2537516
## NOX       -0.0803564 -0.08906087  0.5840831  0.3128615  1.00000000  0.9230345
## SO2       -0.1737525 -0.26637581  0.6185629  0.2537516  0.92303455  1.0000000
## Mortality  0.2925457 -0.58057023  0.5787939  0.5271655  0.56885330  0.5879137
##            Mortality
## Rainfall   0.2925457
## Education -0.5805702
## Popden     0.5787939
## Nonwhite   0.5271655
## NOX        0.5688533
## SO2        0.5879137
## Mortality  1.0000000
\end{verbatim}

\begin{Shaded}
\begin{Highlighting}[]
\NormalTok{melted.my.cor2cov }\OtherTok{\textless{}{-}} \FunctionTok{melt}\NormalTok{(}\FunctionTok{cov2cor}\NormalTok{(my.S))}

\FunctionTok{ggplot}\NormalTok{(}\AttributeTok{data =}\NormalTok{ melted.my.cor2cov, }\FunctionTok{aes}\NormalTok{(}\AttributeTok{x=}\NormalTok{X1, }\AttributeTok{y=}\NormalTok{X2, }\AttributeTok{fill=}\NormalTok{value)) }\SpecialCharTok{+} 
  \FunctionTok{geom\_tile}\NormalTok{() }\SpecialCharTok{+}
  \FunctionTok{geom\_text}\NormalTok{(}\FunctionTok{aes}\NormalTok{(}\AttributeTok{label =} \FunctionTok{round}\NormalTok{(value, }\DecValTok{1}\NormalTok{)))}
\end{Highlighting}
\end{Shaded}

\includegraphics{Project1_covariance_correlation_distance_plots_files/figure-latex/unnamed-chunk-3-1.pdf}

\hypertarget{covariancecorrelation-analysis}{%
\paragraph{Covariance/correlation
analysis}\label{covariancecorrelation-analysis}}

Looking at our correlation plots, a few things really stood out to me.
The first being the negative correlation between education and rainfall.
There is also a strong positive correlation between NOX and SO2. It
would make sense for pollutanats to be holistically more present in
places with more pollution versus those without.

We can also see the positive correlation between NOX and SO2 has on
mortality rates! Theo only negative correlation is education, which is
quite interesting.

\hypertarget{calculating-weighted-distance}{%
\subsubsection{2 - Calculating weighted
distance}\label{calculating-weighted-distance}}

Describe some of the most similar pairs of cities and some of the most
different pairs of cities, giving evidence from the distance matrix.

\begin{Shaded}
\begin{Highlighting}[]
\NormalTok{std }\OtherTok{\textless{}{-}} \FunctionTok{sapply}\NormalTok{(airpol.data.sub, sd) }\CommentTok{\# finding standard deviations of variables}
\NormalTok{airpol.data.std }\OtherTok{\textless{}{-}} \FunctionTok{sweep}\NormalTok{(airpol.data.sub,}\DecValTok{2}\NormalTok{,std,}\AttributeTok{FUN=}\StringTok{"/"}\NormalTok{)  }\CommentTok{\# dividing each variable by its standard deviation}
\NormalTok{dis }\OtherTok{\textless{}{-}} \FunctionTok{dist}\NormalTok{(airpol.data.std) }
\NormalTok{dis.matrix}\OtherTok{\textless{}{-}}\FunctionTok{dist2full}\NormalTok{(dis) }
\FunctionTok{rownames}\NormalTok{(dis.matrix) }\OtherTok{\textless{}{-}}\NormalTok{ city.names}
\FunctionTok{colnames}\NormalTok{(dis.matrix) }\OtherTok{\textless{}{-}}\NormalTok{ city.names}

\NormalTok{my.melted.dist }\OtherTok{\textless{}{-}} \FunctionTok{melt}\NormalTok{(}\FunctionTok{round}\NormalTok{(dis.matrix,}\AttributeTok{digits=}\DecValTok{2}\NormalTok{))}

\FunctionTok{ggplot}\NormalTok{(}\AttributeTok{data =}\NormalTok{ my.melted.dist, }\FunctionTok{aes}\NormalTok{(}\AttributeTok{x=}\NormalTok{X1, }\AttributeTok{y=}\NormalTok{X2, }\AttributeTok{fill=}\NormalTok{value)) }\SpecialCharTok{+} 
  \FunctionTok{geom\_tile}\NormalTok{() }\SpecialCharTok{+}
  \FunctionTok{geom\_text}\NormalTok{(}\FunctionTok{aes}\NormalTok{(}\AttributeTok{label =} \FunctionTok{round}\NormalTok{(value, }\DecValTok{1}\NormalTok{)))}
\end{Highlighting}
\end{Shaded}

\includegraphics{Project1_covariance_correlation_distance_plots_files/figure-latex/unnamed-chunk-4-1.pdf}

\hypertarget{distance-analysis}{%
\paragraph{Distance analysis}\label{distance-analysis}}

We can see that some of the biggest differences are stemming from
Chicago, IL. It differs most from other metro areas like Atlanta, GA and
even Dallas, TX!

Some of the most similar are among the cities in Ohio - Akron, Canton,
Cleveland, and Columbus.

\hypertarget{determining-multivariate-normality}{%
\subsubsection{3 - Determining multivariate
normality}\label{determining-multivariate-normality}}

What is your conclusion based on the plot?

\begin{Shaded}
\begin{Highlighting}[]
\CommentTok{\#Copy the chisplot function into R}
\NormalTok{chisplot }\OtherTok{\textless{}{-}} \ControlFlowTok{function}\NormalTok{(x) \{}
  \ControlFlowTok{if}\NormalTok{ (}\SpecialCharTok{!}\FunctionTok{is.matrix}\NormalTok{(x)) }\FunctionTok{stop}\NormalTok{(}\StringTok{"x is not a matrix"}\NormalTok{)}
  \DocumentationTok{\#\#\# determine dimensions}
\NormalTok{  n }\OtherTok{\textless{}{-}} \FunctionTok{nrow}\NormalTok{(x)}
\NormalTok{  p }\OtherTok{\textless{}{-}} \FunctionTok{ncol}\NormalTok{(x)}
\NormalTok{  xbar }\OtherTok{\textless{}{-}} \FunctionTok{apply}\NormalTok{(x, }\DecValTok{2}\NormalTok{, mean)}
\NormalTok{  S }\OtherTok{\textless{}{-}} \FunctionTok{var}\NormalTok{(x)}
\NormalTok{  S }\OtherTok{\textless{}{-}} \FunctionTok{solve}\NormalTok{(S)}
\NormalTok{  index }\OtherTok{\textless{}{-}}\NormalTok{ (}\DecValTok{1}\SpecialCharTok{:}\NormalTok{n)}\SpecialCharTok{/}\NormalTok{(n}\SpecialCharTok{+}\DecValTok{1}\NormalTok{)}
\NormalTok{  xcent }\OtherTok{\textless{}{-}} \FunctionTok{t}\NormalTok{(}\FunctionTok{t}\NormalTok{(x) }\SpecialCharTok{{-}}\NormalTok{ xbar)}
\NormalTok{  di }\OtherTok{\textless{}{-}} \FunctionTok{apply}\NormalTok{(xcent, }\DecValTok{1}\NormalTok{, }\ControlFlowTok{function}\NormalTok{(x,S) x }\SpecialCharTok{\%*\%}\NormalTok{ S }\SpecialCharTok{\%*\%}\NormalTok{ x,S)}
\NormalTok{  quant }\OtherTok{\textless{}{-}} \FunctionTok{qchisq}\NormalTok{(index,p)}
  \FunctionTok{plot}\NormalTok{(quant, }\FunctionTok{sort}\NormalTok{(di), }\AttributeTok{ylab =} \StringTok{"Ordered distances"}\NormalTok{,}
  \AttributeTok{xlab =} \StringTok{"Chi{-}square quantile"}\NormalTok{, }\AttributeTok{lwd=}\DecValTok{2}\NormalTok{,}\AttributeTok{pch=}\DecValTok{1}\NormalTok{)}
\NormalTok{\}}

\CommentTok{\# Note: You must check if airpol.full has non{-}numerical columns or not. Use only numerical columns for plot!}
\FunctionTok{chisplot}\NormalTok{(}\FunctionTok{as.matrix}\NormalTok{(}\FunctionTok{select\_if}\NormalTok{(airpol.full, is.numeric)))}
\end{Highlighting}
\end{Shaded}

\includegraphics{Project1_covariance_correlation_distance_plots_files/figure-latex/unnamed-chunk-5-1.pdf}

\hypertarget{normality-analysis}{%
\paragraph{Normality analysis}\label{normality-analysis}}

Given the quantile plot above, we can see an exponentially increase in
distance from our mean expected value. This implies that mortality is
not normal across the United States in cities that were measured!

\end{document}
