% Options for packages loaded elsewhere
\PassOptionsToPackage{unicode}{hyperref}
\PassOptionsToPackage{hyphens}{url}
%
\documentclass[
]{article}
\usepackage{amsmath,amssymb}
\usepackage{lmodern}
\usepackage{iftex}
\ifPDFTeX
  \usepackage[T1]{fontenc}
  \usepackage[utf8]{inputenc}
  \usepackage{textcomp} % provide euro and other symbols
\else % if luatex or xetex
  \usepackage{unicode-math}
  \defaultfontfeatures{Scale=MatchLowercase}
  \defaultfontfeatures[\rmfamily]{Ligatures=TeX,Scale=1}
\fi
% Use upquote if available, for straight quotes in verbatim environments
\IfFileExists{upquote.sty}{\usepackage{upquote}}{}
\IfFileExists{microtype.sty}{% use microtype if available
  \usepackage[]{microtype}
  \UseMicrotypeSet[protrusion]{basicmath} % disable protrusion for tt fonts
}{}
\makeatletter
\@ifundefined{KOMAClassName}{% if non-KOMA class
  \IfFileExists{parskip.sty}{%
    \usepackage{parskip}
  }{% else
    \setlength{\parindent}{0pt}
    \setlength{\parskip}{6pt plus 2pt minus 1pt}}
}{% if KOMA class
  \KOMAoptions{parskip=half}}
\makeatother
\usepackage{xcolor}
\usepackage[margin=1in]{geometry}
\usepackage{color}
\usepackage{fancyvrb}
\newcommand{\VerbBar}{|}
\newcommand{\VERB}{\Verb[commandchars=\\\{\}]}
\DefineVerbatimEnvironment{Highlighting}{Verbatim}{commandchars=\\\{\}}
% Add ',fontsize=\small' for more characters per line
\usepackage{framed}
\definecolor{shadecolor}{RGB}{248,248,248}
\newenvironment{Shaded}{\begin{snugshade}}{\end{snugshade}}
\newcommand{\AlertTok}[1]{\textcolor[rgb]{0.94,0.16,0.16}{#1}}
\newcommand{\AnnotationTok}[1]{\textcolor[rgb]{0.56,0.35,0.01}{\textbf{\textit{#1}}}}
\newcommand{\AttributeTok}[1]{\textcolor[rgb]{0.77,0.63,0.00}{#1}}
\newcommand{\BaseNTok}[1]{\textcolor[rgb]{0.00,0.00,0.81}{#1}}
\newcommand{\BuiltInTok}[1]{#1}
\newcommand{\CharTok}[1]{\textcolor[rgb]{0.31,0.60,0.02}{#1}}
\newcommand{\CommentTok}[1]{\textcolor[rgb]{0.56,0.35,0.01}{\textit{#1}}}
\newcommand{\CommentVarTok}[1]{\textcolor[rgb]{0.56,0.35,0.01}{\textbf{\textit{#1}}}}
\newcommand{\ConstantTok}[1]{\textcolor[rgb]{0.00,0.00,0.00}{#1}}
\newcommand{\ControlFlowTok}[1]{\textcolor[rgb]{0.13,0.29,0.53}{\textbf{#1}}}
\newcommand{\DataTypeTok}[1]{\textcolor[rgb]{0.13,0.29,0.53}{#1}}
\newcommand{\DecValTok}[1]{\textcolor[rgb]{0.00,0.00,0.81}{#1}}
\newcommand{\DocumentationTok}[1]{\textcolor[rgb]{0.56,0.35,0.01}{\textbf{\textit{#1}}}}
\newcommand{\ErrorTok}[1]{\textcolor[rgb]{0.64,0.00,0.00}{\textbf{#1}}}
\newcommand{\ExtensionTok}[1]{#1}
\newcommand{\FloatTok}[1]{\textcolor[rgb]{0.00,0.00,0.81}{#1}}
\newcommand{\FunctionTok}[1]{\textcolor[rgb]{0.00,0.00,0.00}{#1}}
\newcommand{\ImportTok}[1]{#1}
\newcommand{\InformationTok}[1]{\textcolor[rgb]{0.56,0.35,0.01}{\textbf{\textit{#1}}}}
\newcommand{\KeywordTok}[1]{\textcolor[rgb]{0.13,0.29,0.53}{\textbf{#1}}}
\newcommand{\NormalTok}[1]{#1}
\newcommand{\OperatorTok}[1]{\textcolor[rgb]{0.81,0.36,0.00}{\textbf{#1}}}
\newcommand{\OtherTok}[1]{\textcolor[rgb]{0.56,0.35,0.01}{#1}}
\newcommand{\PreprocessorTok}[1]{\textcolor[rgb]{0.56,0.35,0.01}{\textit{#1}}}
\newcommand{\RegionMarkerTok}[1]{#1}
\newcommand{\SpecialCharTok}[1]{\textcolor[rgb]{0.00,0.00,0.00}{#1}}
\newcommand{\SpecialStringTok}[1]{\textcolor[rgb]{0.31,0.60,0.02}{#1}}
\newcommand{\StringTok}[1]{\textcolor[rgb]{0.31,0.60,0.02}{#1}}
\newcommand{\VariableTok}[1]{\textcolor[rgb]{0.00,0.00,0.00}{#1}}
\newcommand{\VerbatimStringTok}[1]{\textcolor[rgb]{0.31,0.60,0.02}{#1}}
\newcommand{\WarningTok}[1]{\textcolor[rgb]{0.56,0.35,0.01}{\textbf{\textit{#1}}}}
\usepackage{graphicx}
\makeatletter
\def\maxwidth{\ifdim\Gin@nat@width>\linewidth\linewidth\else\Gin@nat@width\fi}
\def\maxheight{\ifdim\Gin@nat@height>\textheight\textheight\else\Gin@nat@height\fi}
\makeatother
% Scale images if necessary, so that they will not overflow the page
% margins by default, and it is still possible to overwrite the defaults
% using explicit options in \includegraphics[width, height, ...]{}
\setkeys{Gin}{width=\maxwidth,height=\maxheight,keepaspectratio}
% Set default figure placement to htbp
\makeatletter
\def\fps@figure{htbp}
\makeatother
\setlength{\emergencystretch}{3em} % prevent overfull lines
\providecommand{\tightlist}{%
  \setlength{\itemsep}{0pt}\setlength{\parskip}{0pt}}
\setcounter{secnumdepth}{-\maxdimen} % remove section numbering
\ifLuaTeX
  \usepackage{selnolig}  % disable illegal ligatures
\fi
\IfFileExists{bookmark.sty}{\usepackage{bookmark}}{\usepackage{hyperref}}
\IfFileExists{xurl.sty}{\usepackage{xurl}}{} % add URL line breaks if available
\urlstyle{same} % disable monospaced font for URLs
\hypersetup{
  pdftitle={DA410\_Project2\_MattGraham},
  hidelinks,
  pdfcreator={LaTeX via pandoc}}

\title{DA410\_Project2\_MattGraham}
\author{}
\date{\vspace{-2.5em}}

\begin{document}
\maketitle

This is our first project, analyzing air pollution, mortality rates, and
relevant parameters.

\begin{Shaded}
\begin{Highlighting}[]
\FunctionTok{library}\NormalTok{(nnspat)  }\CommentTok{\# used for dist2full()}
\FunctionTok{library}\NormalTok{(}\StringTok{"dplyr"}\NormalTok{)  }\CommentTok{\# used to select numeric datatypes}
\FunctionTok{library}\NormalTok{(}\StringTok{"ggplot2"}\NormalTok{)}
\FunctionTok{library}\NormalTok{(reshape)  }\CommentTok{\# used for melting matricies}
\end{Highlighting}
\end{Shaded}

\hypertarget{part-1}{%
\subsection{Part 1}\label{part-1}}

\hypertarget{scores-by-sex}{%
\paragraph{Scores by sex}\label{scores-by-sex}}

\begin{Shaded}
\begin{Highlighting}[]
\NormalTok{testdata }\OtherTok{\textless{}{-}} \FunctionTok{read.table}\NormalTok{(}\StringTok{"C:/mattgraham93.github.io/school/22\_3\_DA410/data/testscoredata.txt"}\NormalTok{, }\AttributeTok{header=}\ConstantTok{TRUE}\NormalTok{)}
\NormalTok{testdata.noIDs }\OtherTok{\textless{}{-}}\NormalTok{ testdata[,}\SpecialCharTok{{-}}\DecValTok{1}\NormalTok{]  }\CommentTok{\#to remove the ID numbers}
\FunctionTok{as.data.frame}\NormalTok{(testdata.noIDs)}
\end{Highlighting}
\end{Shaded}

\begin{verbatim}
##      math reading  sex
## 1   83.16   79.67  boy
## 2  102.51  101.13  boy
## 3   81.63   80.53  boy
## 4   88.25   84.58  boy
## 5   81.47   76.52  boy
## 6   87.19   84.70  boy
## 7   88.66   85.86  boy
## 8   79.35   81.03  boy
## 9   83.35   80.44  boy
## 10  86.58   84.67  boy
## 11  81.73   80.71  boy
## 12  85.00   81.32  boy
## 13  85.23   81.31  boy
## 14  80.30   79.37  boy
## 15  81.18   79.65  boy
## 16  88.41   85.85  boy
## 17  90.80   88.81  boy
## 18  81.68   79.71  boy
## 19  82.22   79.81  boy
## 20  78.21   74.20  boy
## 21  72.64   69.13  boy
## 22  84.61   83.05  boy
## 23  82.06   82.12  boy
## 24  87.01   84.62  boy
## 25  86.25   85.45  boy
## 26  77.05   74.03  boy
## 27  90.76   87.13  boy
## 28  81.39   78.53  boy
## 29  81.20   79.73  boy
## 30  83.07   79.94  boy
## 31  76.99   75.74 girl
## 32  83.32   81.40 girl
## 33  75.37   78.26 girl
## 34  84.81   83.93 girl
## 35  81.61   79.58 girl
## 36  76.08   78.18 girl
## 37  84.43   86.48 girl
## 38  82.29   83.16 girl
## 39  81.91   81.88 girl
## 40  97.85   97.01 girl
## 41  75.96   75.72 girl
## 42  82.47   80.84 girl
## 43  78.43   75.01 girl
## 44  82.89   82.92 girl
## 45  86.26   84.84 girl
## 46  88.48   87.90 girl
## 47  82.47   82.29 girl
## 48  87.24   87.45 girl
## 49  79.72   79.75 girl
## 50  87.52   85.44 girl
## 51  84.73   82.24 girl
## 52  77.15   77.63 girl
## 53  85.33   81.96 girl
## 54  80.58   81.67 girl
## 55  88.70   87.57 girl
## 56  79.20   77.14 girl
## 57  91.84   91.55 girl
## 58  81.07   77.01 girl
## 59  88.15   88.16 girl
## 60  76.98   75.65 girl
## 61  79.27   80.33 girl
## 62  85.70   84.27 girl
\end{verbatim}

\hypertarget{hotellings-test}{%
\subsubsection{Hotelling's test}\label{hotellings-test}}

\begin{Shaded}
\begin{Highlighting}[]
\FunctionTok{summary}\NormalTok{(}\FunctionTok{manova}\NormalTok{(}\FunctionTok{cbind}\NormalTok{(math, reading) }\SpecialCharTok{\textasciitilde{}}\NormalTok{sex, }\AttributeTok{data=}\NormalTok{testdata.noIDs), }\AttributeTok{test=}\StringTok{"Hotelling"}\NormalTok{)}
\end{Highlighting}
\end{Shaded}

\begin{verbatim}
##           Df Hotelling-Lawley approx F num Df den Df    Pr(>F)    
## sex        1          0.30593   9.0249      2     59 0.0003805 ***
## Residuals 60                                                      
## ---
## Signif. codes:  0 '***' 0.001 '**' 0.01 '*' 0.05 '.' 0.1 ' ' 1
\end{verbatim}

\hypertarget{hotellings-analysis}{%
\paragraph{Hotelling's Analysis}\label{hotellings-analysis}}

At alpha = 0.05 and p-value \textless{} 0.05, we can conclude there is
sufficient evidence to state there are differences between mean math and
reading scores between the recorded sexes.

\hypertarget{part-2}{%
\subsection{Part 2}\label{part-2}}

Suppose we have gathered the following data on female athletes in three
sports. The measurements we have made are the athletes' heights and
vertical jumps, both in inches. The data are listed as (height, jump) as
follows:

\begin{Shaded}
\begin{Highlighting}[]
\NormalTok{sport }\OtherTok{\textless{}{-}} \FunctionTok{c}\NormalTok{(}\StringTok{\textquotesingle{}B\textquotesingle{}}\NormalTok{,}\StringTok{\textquotesingle{}B\textquotesingle{}}\NormalTok{,}\StringTok{\textquotesingle{}B\textquotesingle{}}\NormalTok{,}\StringTok{\textquotesingle{}B\textquotesingle{}}\NormalTok{,}\StringTok{\textquotesingle{}B\textquotesingle{}}\NormalTok{,}\StringTok{\textquotesingle{}T\textquotesingle{}}\NormalTok{,}\StringTok{\textquotesingle{}T\textquotesingle{}}\NormalTok{,}\StringTok{\textquotesingle{}T\textquotesingle{}}\NormalTok{,}\StringTok{\textquotesingle{}T\textquotesingle{}}\NormalTok{,}\StringTok{\textquotesingle{}S\textquotesingle{}}\NormalTok{,}\StringTok{\textquotesingle{}S\textquotesingle{}}\NormalTok{,}\StringTok{\textquotesingle{}S\textquotesingle{}}\NormalTok{,}\StringTok{\textquotesingle{}S\textquotesingle{}}\NormalTok{,}\StringTok{\textquotesingle{}S\textquotesingle{}}\NormalTok{,}\StringTok{\textquotesingle{}S\textquotesingle{}}\NormalTok{) }
\NormalTok{height }\OtherTok{\textless{}{-}} \FunctionTok{c}\NormalTok{(}\DecValTok{66}\NormalTok{,}\DecValTok{65}\NormalTok{,}\DecValTok{68}\NormalTok{,}\DecValTok{64}\NormalTok{,}\DecValTok{67}\NormalTok{,}\DecValTok{63}\NormalTok{,}\DecValTok{61}\NormalTok{,}\DecValTok{62}\NormalTok{,}\DecValTok{60}\NormalTok{,}\DecValTok{62}\NormalTok{,}\DecValTok{65}\NormalTok{,}\DecValTok{63}\NormalTok{,}\DecValTok{62}\NormalTok{,}\FloatTok{63.5}\NormalTok{,}\DecValTok{66}\NormalTok{) }
\NormalTok{jump}\OtherTok{\textless{}{-}}\FunctionTok{c}\NormalTok{(}\DecValTok{27}\NormalTok{,}\DecValTok{29}\NormalTok{,}\DecValTok{26}\NormalTok{,}\DecValTok{29}\NormalTok{,}\DecValTok{29}\NormalTok{,}\DecValTok{23}\NormalTok{,}\DecValTok{26}\NormalTok{,}\DecValTok{23}\NormalTok{,}\DecValTok{26}\NormalTok{,}\DecValTok{23}\NormalTok{,}\DecValTok{21}\NormalTok{,}\DecValTok{21}\NormalTok{,}\DecValTok{23}\NormalTok{,}\DecValTok{22}\NormalTok{,}\FloatTok{21.5}\NormalTok{)}

\NormalTok{sports }\OtherTok{\textless{}{-}} \FunctionTok{as.data.frame}\NormalTok{(}\FunctionTok{cbind}\NormalTok{(sport, jump, height))}
\NormalTok{sports}\SpecialCharTok{$}\NormalTok{jump }\OtherTok{\textless{}{-}} \FunctionTok{as.numeric}\NormalTok{(sports}\SpecialCharTok{$}\NormalTok{jump)}
\NormalTok{sports}\SpecialCharTok{$}\NormalTok{height }\OtherTok{\textless{}{-}} \FunctionTok{as.numeric}\NormalTok{(sports}\SpecialCharTok{$}\NormalTok{height)}

\NormalTok{sports}
\end{Highlighting}
\end{Shaded}

\begin{verbatim}
##    sport jump height
## 1      B 27.0   66.0
## 2      B 29.0   65.0
## 3      B 26.0   68.0
## 4      B 29.0   64.0
## 5      B 29.0   67.0
## 6      T 23.0   63.0
## 7      T 26.0   61.0
## 8      T 23.0   62.0
## 9      T 26.0   60.0
## 10     S 23.0   62.0
## 11     S 21.0   65.0
## 12     S 21.0   63.0
## 13     S 23.0   62.0
## 14     S 22.0   63.5
## 15     S 21.5   66.0
\end{verbatim}

\hypertarget{wilks-lambda-test}{%
\subsubsection{Wilks' Lambda test}\label{wilks-lambda-test}}

\begin{Shaded}
\begin{Highlighting}[]
\FunctionTok{summary}\NormalTok{(}\FunctionTok{manova}\NormalTok{(}\FunctionTok{cbind}\NormalTok{(height, jump) }\SpecialCharTok{\textasciitilde{}}\NormalTok{ sport), }\AttributeTok{data=}\NormalTok{sports, }\AttributeTok{test=}\StringTok{"Wilks"}\NormalTok{)}
\end{Highlighting}
\end{Shaded}

\begin{verbatim}
##           Df    Wilks approx F num Df den Df    Pr(>F)    
## sport      2 0.035879   23.536      4     22 1.117e-07 ***
## Residuals 12                                              
## ---
## Signif. codes:  0 '***' 0.001 '**' 0.01 '*' 0.05 '.' 0.1 ' ' 1
\end{verbatim}

\hypertarget{assumption-check}{%
\subsubsection{Assumption check}\label{assumption-check}}

\begin{Shaded}
\begin{Highlighting}[]
\NormalTok{sport.manova }\OtherTok{\textless{}{-}} \FunctionTok{manova}\NormalTok{(}\FunctionTok{cbind}\NormalTok{(height, jump) }\SpecialCharTok{\textasciitilde{}}\NormalTok{ sport)}
  
\NormalTok{chisplot }\OtherTok{\textless{}{-}} \ControlFlowTok{function}\NormalTok{(x) \{ }
    \ControlFlowTok{if}\NormalTok{ (}\SpecialCharTok{!}\FunctionTok{is.matrix}\NormalTok{(x)) }\FunctionTok{stop}\NormalTok{(}\StringTok{"x is not a matrix"}\NormalTok{) }
    \DocumentationTok{\#\#\# determine dimensions }
\NormalTok{    n }\OtherTok{\textless{}{-}} \FunctionTok{nrow}\NormalTok{(x) }
\NormalTok{    p }\OtherTok{\textless{}{-}} \FunctionTok{ncol}\NormalTok{(x) }
\NormalTok{    xbar }\OtherTok{\textless{}{-}} \FunctionTok{apply}\NormalTok{(x, }\DecValTok{2}\NormalTok{, mean) }
\NormalTok{    S }\OtherTok{\textless{}{-}} \FunctionTok{var}\NormalTok{(x) }
\NormalTok{    S }\OtherTok{\textless{}{-}} \FunctionTok{solve}\NormalTok{(S) }
\NormalTok{    index }\OtherTok{\textless{}{-}}\NormalTok{ (}\DecValTok{1}\SpecialCharTok{:}\NormalTok{n)}\SpecialCharTok{/}\NormalTok{(n}\SpecialCharTok{+}\DecValTok{1}\NormalTok{) }
\NormalTok{    xcent }\OtherTok{\textless{}{-}} \FunctionTok{t}\NormalTok{(}\FunctionTok{t}\NormalTok{(x) }\SpecialCharTok{{-}}\NormalTok{ xbar) }
\NormalTok{    di }\OtherTok{\textless{}{-}} \FunctionTok{apply}\NormalTok{(xcent, }\DecValTok{1}\NormalTok{, }\ControlFlowTok{function}\NormalTok{(x,S) x }\SpecialCharTok{\%*\%}\NormalTok{ S }\SpecialCharTok{\%*\%}\NormalTok{ x,S) }
\NormalTok{    quant }\OtherTok{\textless{}{-}} \FunctionTok{qchisq}\NormalTok{(index,p) }
    \FunctionTok{plot}\NormalTok{(quant, }\FunctionTok{sort}\NormalTok{(di), }\AttributeTok{ylab =} \StringTok{"Ordered distances"}\NormalTok{, }
         \AttributeTok{xlab =} \StringTok{"Chi{-}square quantile"}\NormalTok{, }\AttributeTok{lwd=}\DecValTok{2}\NormalTok{,}\AttributeTok{pch=}\DecValTok{1}\NormalTok{) }
\NormalTok{\}}

\FunctionTok{chisplot}\NormalTok{(}\FunctionTok{residuals}\NormalTok{(sport.manova))}
\end{Highlighting}
\end{Shaded}

\includegraphics{DA410_Project2_MattGraham_files/figure-latex/unnamed-chunk-5-1.pdf}

\hypertarget{normality-analysis}{%
\paragraph{Normality analysis}\label{normality-analysis}}

There is a bit of a sine pattern on our plot. This does make me
second-guess the assumption of our tests. However, As this is a
mostly-linear line, we can assume we have met all our assumptions and
can continue stating there are differences between recorded sexes.

\hypertarget{mean-analysis}{%
\subsubsection{Mean analysis}\label{mean-analysis}}

\begin{Shaded}
\begin{Highlighting}[]
\NormalTok{my.n }\OtherTok{\textless{}{-}} \FunctionTok{nrow}\NormalTok{(sports)  }\CommentTok{\# number of individuals }
 
\CommentTok{\# Sample mean vectors for the SAT scores data: }
\FunctionTok{as.data.frame}\NormalTok{(}
\NormalTok{  sports }\SpecialCharTok{\%\textgreater{}\%} 
    \FunctionTok{group\_by}\NormalTok{(sport) }\SpecialCharTok{\%\textgreater{}\%}
    \FunctionTok{summarise\_at}\NormalTok{(}\FunctionTok{vars}\NormalTok{(}\StringTok{"jump"}\NormalTok{, }\StringTok{"height"}\NormalTok{), mean)}
\NormalTok{)}
\end{Highlighting}
\end{Shaded}

\begin{verbatim}
##   sport     jump   height
## 1     B 28.00000 66.00000
## 2     S 21.91667 63.58333
## 3     T 24.50000 61.50000
\end{verbatim}

Above, we can see the differences in our height vs.~jump averages. We
can note our softball players are between track and basketball players
in terms of height. They are, however, lowest when jumping. Meanwhile,
the inverse of that statement is true for our shorest track folks. Our
basketball players are both the tallest and jump the highest.

\end{document}
